

% Define \NL (newline) so LaTeX goes to the next output line.
% Just doing \\ complains
%     ! LaTeX Error: There's no line here to end.
% \mbox{} is an empty math box.
\newcommand{\NL}{\mbox{}\\}



% To keep track of variables used in the document
%    for optional printing when in debug mode
\makeatletter
    \newcommand*\paperVariables{}
    \newcommand*\paperSortVars{}
    \newcommand*\storeVariables[2]
    {\ifx\paperVariables\empty
        \def\paperVariables{{#1,#2}}%
    \else
        \g@addto@macro\paperVariables{, {#1,#2}}%
    \fi
    }
    \makeatother
    

% Use \varDef instead of \newCommand to track variables for debug mode
%       #1: latex command to call to create the variable (e.g. \myAngle)
%       #2: symbol to use (e.g. a, A, \alpha)
%       #3: (optional) font class (e.g. \mathbb )
\NewDocumentCommand{\varDef}{m m o} {
    \IfValueTF{#3}{\newcommand{#1}{#3{#2}}}{\newcommand{#1}{#2}}
    % Only store variables if needed for debug mode
    \ifthenelse{\equal{debug}{\ZZbuildmode}} {
        \storeVariables{#1}{#2}}{}
    }


% Vector formatting
\newcommand{\vectorFmt}[1]{\pmb{#1}}
\newcommand{\unitVecFmt}[1]{\hat{#1}}
\newcommand{\sub}[2]{{#1}_{#2}}
\newcommand{\upRight}[2]{{#1}^{#2}}
\newcommand{\upLeft}[2]{{}^{#2} {#1}}