
% Define useful variables of the form:
%    \vardef{\command}{var-name}[\format] 

\varDef{\zNd}{z}
\varDef{\xNd}{x}
\varDef{\yNd}{y}
\varDef{\xDim}{X}
\varDef{\yDim}{Y}
\varDef{\zDim}{Z}
\varDef{\angleTwo}{\beta}
\varDef{\angleOne}{\alpha}
\varDef{\angleFour}{\Pi}
\varDef{\angleThree}{\Gamma}
\varDef{\otherCvar}{C}
\varDef{\nBody}{N}

% Math
\varDef{\Real}{R}[\mathbb]
\varDef{\Complex}{C}[\mathbb]

% Stats
\varDef{\Normal}{N}[\mathcal]
\varDef{\Uniform}{U}[\mathcal]
\varDef{\ExpVal}{E}[\mathbb]

% Dynamical Systems
\varDef{\lagrangian}{L}[\mathcal]
\varDef{\hamiltonian}{H}[\mathcal]

% To show how formatting is handled
\varDef{\vectorFormat}{x}[\vectorFmt]
\varDef{\unitVectorFormat}{x}[\unitVecFmt]
% How about superscripts?
\varDef{\xSquared}{x}^{2}
\varDef{\xSubTwo}{x}_{2}
\varDef{\xSubTwoSquared}{x}_{2}^{2}
\varDef{\aVectorSquared}{y}[\vectorFmt]^{2}

% Create / sort database of paper variables for debugging
% \DTLnewdb{varDB}
% \foreach \cmd [count=\i] in \paperVariables { 
%     \DTLnewrow{varDB}
%     \dtlexpandnewvalue
%     \DTLnewdbentry{varDB}{Variable}{\cmd}
%     \DTLnewdbentry{varDB}{SortVar}{x}
%     \DTLnewdbentry{varDB}{VarCmd}{\expandafter\string\cmd}
% }
% Sort list
% \DTLsort{SortVar}{varDB}

% Database part is inspired from:
%  https://tex.stackexchange.com/questions/121489/alphabetically-display-the-items-in-itemize
