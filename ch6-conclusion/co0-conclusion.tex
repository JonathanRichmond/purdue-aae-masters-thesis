\chapter{CONCLUDING REMARKS}
For humanity to develop and maintain a constant presence at Mars and other deep-space locations, a
comprehensive understanding of multi-body dynamical systems theory and departure dynamics within
cislunar space is critical. This investigation develops a methodology for designing transfers from
unstable cislunar periodic orbits to deep-space targets utilizing invariant manifolds in the CR3BP.
By applying dynamical systems theory, this design approach links multiple CR3BP systems for
end-to-end transfers that exist in families of solutions, allowing for more flexible mission
designs. Additionally, in this investigation, various unstable cislunar periodic orbit families are
analyzed in the context of these interplanetary transfers to evaluate their departure
characteristics. The result is an improved understanding of invariant manifold departure behavior
in the Earth-Moon CR3BP, as well as a catalog of Mars transfer tradespaces. The end-to-end transfer
maneuver costs in this investigation are consistent with other lower-energy transfers in previous
literature and better than traditional approaches, although the time-of-flight increases
significantly. This section summarizes the contributions of this investigation and provides
several recommendations for future work.

\section{Investigation Summary}
\subsection{Low-Energy Cislunar-to-Mars Transfer Design Methodology}
Unstable periodic orbits in the CR3BP possess unstable and stable invariant manifolds that
asymptotically depart from or arrive onto the orbit and arcs from these manifolds are utilized to
construct ballistic transfers starting or terminating at these orbits. However, in the context of
interplanetary and deep-space transfers, the energy gaps between invariant manifolds of different
systems prevent direct connections between these manifolds, dictating the need for bridging arcs
or alternative strategies. The end-to-end cislunar-to-Mars transfer design methodology developed in
this investigation patches two planetary manifold arcs together with a heliocentric Keplerian
bridging arc employing the MMAT approach. Since this bridging leg of the trajectory is sufficiently
far away from both the Earth and Mars, the solar gravity is the dominating force considered and is
adequately modeled by the relative 2BP, leading to a semi-analytical method for computing this part
of the transfer. This approach is also able to incorporate the true orbital planes of Earth and
Mars, resulting in a higher model fidelity and robust framework for a transfer design to bridge
planetary systems. 

In addition to the MMAT methodology, some end-to-end transfers constructed in this investigation
stage in an intermediate $L_{2}$ halo orbit in the Sun-Earth CR3BP. Those solutions rely on
near-ballistic connections between unstable Earth-Moon and stable Sun-Earth invariant manifold arcs
for low-cost transfers in a blended CR3BP model. Solutions that instead directly depart the system
without connecting to a staging orbit still rely on this blended model, transitioning from one
CR3BP system to the other at the edge of the Moon sphere of influence. The successful meshing of
the blended CR3BP and patched 2BP-CR3BP dynamical models facilitates the end-to-end transfers
between Earth-Moon and Sun-Mars CR3BP periodic orbits.

Another major benefit of this new transfer methodology is that the transfers exist in families of
solutions. Through the exploitation of dynamical systems theory, there are a variety of ways to
continue the family of solutions. Each solution tradespace for the orbits employed in this
investigation, all of which are supplied in Appendix A, shows families of transfer solutions
continued by varying either the unstable departure manifold or the initial departure epoch. New
families could also be generated by varying the stable arrival manifold onto the Sun-Mars halo
orbit or the Jacobi constant of the intermediate staging orbit. By not relying on individual point
solutions that have to be recomputed whenever a parameter changes, the transfer methodology
developed in this investigation offers improved mission design flexibility while also providing a
broader view of the design tradespace.

\subsection{Intermediate Sun-Earth Staging Halo Orbits}
Since unstable invariant manifolds asymptotically depart from periodic orbits, the cislunar
manifold arcs take a long time to depart from the Earth-Moon system. Once they do, these manifold
arcs are propagated under the Sun-Earth dynamics until they leave the Earth SoI. In some cases, the
unstable invariant manifolds from Sun-Earth periodic orbits depart the Earth region with more
desirable characteristics. Consequently, for each cislunar departure orbit analyzed in this
investigation, transfers with direct manifold departures from the system are compared to those that
stage in an intermediate Sun-Earth $L_{2}$ northern halo orbit. This comparative analysis
identifies the optimal system departure strategy for each cislunar orbit.

In the context of the end-to-end transfer methodology developed in this investigation, it appears
that the direct departure transfers perform better overall than those that utilize a staging orbit
in terms of total maneuver $\Delta v$ cost and TOF. While staging orbit transfers for a few of the
selected orbits achieve a slightly lower average $\Delta v$, the direct transfers have lower
times-of-flight in every case analyzed. The TOF decrease can be on the order of $1$ year compared
to around $0.1$ km/s for maneuver costs. This conclusion is further reinforced by the cost function
analysis of the transfers, where all of the direct transfer families have lower cost function
values than those with staging orbits. Consequently, in this investigation, the transfers that
directly depart the Earth system appear to outperform those that stage in an intermediate Sun-Earth
halo orbit in terms of TOF and usually maneuver cost as well.

\subsection{Cislunar Departure Characteristics}
Each cislunar departure orbit included in this investigation provides families of transfer
solutions, evaluated in a tradespace between total maneuver $\Delta v$ and total TOF. For each
tradespace, a cost function is applied to determine the ten lowest-cost orbits in each family
(staging orbit and direct) based on mission design preferences. In this investigation, parameters
are selected such that decreases in TOF are valued more highly than decreases in $\Delta v$. The
characteristics of these ten lowest-cost transfers are compared between the different cislunar
departure orbits to identify characteristics across families and energy levels.

The timing and placement of the second MMAT maneuver are crucial in minimizing both the total
$\Delta v$ maneuver cost and TOF for these interplanetary transfers. For both types of transfers
developed in this investigation, the location of the second MMAT maneuver, which includes the
inclination change, is the dominating factor for the transfer total $\Delta v$ cost. The closer
this burn occurs to the periapsis of the heliocentric bridge conic arc, the lower the maneuver
cost. Additionally, in the case of transfers that directly depart the Earth vicinity, unstable
invariant manifold arcs that exit near the Sun-Earth $L_{2}$ point provide the minimum-$\Delta v$
solutions in the family. On the other hand, the main contributing factor for lowering total
transfer TOF is the relative phasing of the Keplerian conic arcs. When the true anomaly of the
arrival conic arc at the Mars SoI intersection is just after the intersection of the bridge and
arrival conic arcs, this minimizes the arrival conic arc TOF and consequently the total transfer
TOF. The minimum-TOF solutions in the families occur when the invariant manifold arcs exit near the
Sun-Earth $L_{1}$ Lagrange point; however, the lowest-cost transfers that balance the two
parameters still exit from $L_{2}$.

The selection of cislunar departure periodic orbit significantly influences both the maneuver cost
and TOF for interplanetary transfers. In terms of maneuver cost, the $L_{1}$ halo and vertical
orbit families, as well as the $L_{2}$ axial orbit family, provide the best options for staging
orbit transfers. For the transfers with direct departures, the $L_{1}$ Lyapunov and halo orbit
families result in the lowest-$\Delta v$ transfer options. The $L_{1}$ family departures leverage
close passes by the Moon to lessen the energy gap between invariant manifold arcs of two planetary
systems. When it comes to the transfer TOF, $L_{2}$ Lyapunov and halo orbit families perform the
best for staging orbit transfers, while $L_{2}$ halo, axial, and butterfly orbit families have the
fastest direct transfers. The $L_{2}$ orbit families often have invariant manifolds that depart the
Earth-Moon system faster and extend further than the $L_{1}$ orbit families. Applying the cost
function to find a balance between $\Delta v$ and TOF, this investigation identifies that transfers
from the $L_{2}$ Lyapunov, halo, and vertical orbit families with direct departures perform well at
various Jacobi constant levels. The best cislunar departure orbit identified in this investigation
is the $3.13$ $L_{2}$ Lyapunov orbit, whose lowest-cost transfers have an average maneuver cost of
$5.282$ km/s and a TOF of $3.92$ years. This analysis underscores the importance of selecting
appropriate cislunar departure orbits to achieve efficient deep-space transfers with a balance
between maneuver cost and TOF.

Employing the two interplanetary transfer methodologies developed in this investigation, all of the
included cislunar departure orbits have staging orbit and direct transfers with total maneuver
costs that are up to $1$ km/s less than the traditional Earth-Mars interplanetary transfer methods.
And while the times-of-flight of these transfers are significantly longer than traditional
interplanetary transfers due to the asymptotic nature of invariant manifolds, the transfers with
direct departures save up to $1$ year compared to the staging orbit transfers. The transfers
developed in this investigation confirm that invariant manifold arcs can be utilized to decrease
interplanetary transfer maneuver costs and highlight a few cislunar orbit families with desirable
departure characteristics.

\section{Recommendations for Future Work}
With increasing interest in interplanetary missions and the application of dynamic systems theory
in multi-body trajectory design, there are many potential avenues of future research to build off
the work done in this investigation. A few promising options are presented here:
\begin{itemize}
    \item   \textbf{Continue this analysis with cislunar departure orbits in a broader Jacobi
            constant range and transfers to other deep-space targets.}

            In the current investigation, cislunar departure orbits were limited to a Jacobi
            constant range of $2.98$-$3.13$. Many potentially useful unstable orbits and orbit
            families in the Earth-Moon CR3BP exist at Jacobi constant values outside of that range.
            For example, there are many families of resonant and period-multiplying orbits that
            were not examined in this investigation. A complete analysis of cislunar departure
            dynamics from unstable orbits would require an extension of the current analysis to
            these other orbits and families. This investigation also limited its scope by selecting
            a Sun-Mars $L_{1}$ northern halo orbit as its arrival destination. The methodologies
            developed are applicable for any unstable CR3BP arrival periodic orbit and should be
            verified with other arrival orbits and CR3BP deep-space target systems such as Venus or
            Jupiter.
    \item   \textbf{Explore cislunar departure dynamics from stable CR3BP orbits.}
    
            Unstable CR3BP departure and arrival periodic orbits are exclusively employed in this
            investigation to exploit their invariant manifolds for ballistic departures and
            arrivals. Unfortunately, the methodologies developed do not apply to stable CR3BP
            orbits due to their lack of these manifolds so other dynamical systems theory
            techniques are required. The utilization of impulsive maneuvers along the most
            stretching directions of a periodic orbit is a promising approach for stable orbit
            departure and could be applied as an alternative to invariant manifold arcs with the
            MMAT method. Muralidharan and Howell demonstrate the usefulness of such an approach
            with some applications within cislunar space\cite{Muralidharan:2022}. This approach
            opens alternative avenues for low-energy deep-space trajectory design by facilitating
            efficient departure from stable CR3BP orbits.
    \item   \textbf{Investigate strategies for incorporating dynamical systems theory and invariant
            manifolds into other interplanetary transfer design methodologies and optimization.}

            This investigation employed a combination of near-ballistic Earth-Moon to Sun-Earth
            transfers and the MMAT methodology to design interplanetary transfers, but this
            technique is not the only way to incorporate multi-body dynamical systems theory into
            deep-space trajectory design. Many other strategies exist that incorporate intentional
            Earth or Moon flybys or different maneuver counts and placements. It is possible that
            some of these methodologies could offer transfers with lower times-of-flight than those
            computed in this investigation. The transfers presented here, while they exist in
            families of solutions, are not optimized. Consequently, an exploration of other
            transfer methodologies that incorporates trajectory optimization is necessary to
            continue to improve interplanetary mission design.
    \item   \textbf{Employ other dynamical models to confirm the results of this investigation and
            further explore cislunar departure dynamics.}

            Since the gravitational effects of the Earth, Moon, and Sun are all considered at
            various stages in this investigation, depending on the dynamical model being applied,
            it would be beneficial to incorporate a 4-body model such as the BCR4BP to efficiently
            represent these dynamics. There is precedent for applying this model to the exploration
            of cislunar departure dynamics, and it would also simplify the construction of
            transfers between the Earth-Moon and Sun-Earth systems, as well as better facilitate
            the relative phasing between the celestial bodies\cite{Boudad:2021,Boudad:2022}. The
            inclusion of the solar gravitational influence in the cislunar region would also aid in
            the departure from stable cislunar orbits through pseudo-manifolds that stem from the
            inclusion of the Sun in the Earth-Moon dynamical model. Finally, the results of this
            investigation, as well as all of those proposed in this future work section, should be
            validated in a high-fidelity ephemeris force model. This validation will ensure that
            the transfer geometries persist under a more accurate representation of the dynamical
            regime while also providing insight into feasible launch dates and windows in the near
            future for these types of deep-space transfers. Transition to higher-fidelity dynamical
            models is the necessary next step for these transfers to be applied to real mission
            scenarios.
\end{itemize}
