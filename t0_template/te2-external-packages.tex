
% -------------------- From Template -------------------

% So "_" will work in URLs when using BibTeX.
\usepackage[T1]{fontenc}

% The mathtools package
% (see http://mirror.utexas.edu/ctan/macros/latex/required/amsmath/amsmath.pdf)
% loads the amsmath package which defines the
%     align
%     align*
%     alignat
%     alignat*
%     equation
%     equation*
%     flalign
%     flalign*
%     gather
%     gather*
%     multitaper
%     multitaper*
%     split
% environments and extends amsmath by defining many other commands.
% See
%     https://ctan.org/pkg/amsmath
% for information about amsmath and
%     http://ctan.math.washington.edu/tex-archive/macros/latex/contrib/mathtools/mathtools.pdf
% for information about mathtools.
\usepackage{mathtools}

% Define \FigureDash.
% \FigureDash is a dash the width of a digit in the current font.
\usepackage{pa-figure-dash}

% For PurdueThesis, PuTh, TeX, LaTeX, METAFONT, METAPOST, etc. related logos.
\usepackage{pa-logos}

% (Or maybe use isomath instead?  -mark  2021-06-20)
% Follow ISO 80000-2:2019
%     o   put e, i, j, and pi in upright font automatically
%     o   use, for example, "\di x" to get "\,mathrm{d}\/x"
% This loads
%     o   amsmath.sty (which is already loaded above)
%     o   mathtools.sty
%     o   upgreek.sty
% Load the package.
\usepackage{pa-mismath}
    % Tell mismath to put e, i, j, and pi in upright font automatically.
    \enumber
    \inumber
    \jnumber
    \pinumber
  % To typeset math italic e, i, j, and pi use
  %     \mathit e
  %     \mathit i
  %     \mathit j
  %     \itpi


% Define \FloatBarrier
\usepackage{placeins}

% For highlighting text using \hl
\usepackage{soul}

% For \sfrac, used to do slanted fractions, similar to, e.g., 1/2, but 1 is small and high and 2 is small and low.
\usepackage{xfrac}

% For typographical conventions stuff including
%     \Emph{...}
%     \First{...}
%     \Keys{...}
%     \Literal{...}
%     \Menu{...}
%     \Place{...}
%     \Shell{...}
% This must be after
%     \usepackage{tikz}
\usepackage{pa-typographic-conventions}


% ----------------- Added Dependencies -----------------

% Generate lorem ipsum 
%    e.g.: \lipsum[1-2]
\usepackage{lipsum}



% cleveref
%   For easier cross-referencing
%   Redefine the subref formatting so that you get:
%     Figure 1.1(a) instead of Figure 1.1a
\usepackage[noabbrev, capitalise]{cleveref}
  \captionsetup[subfigure]{subrefformat=simple,labelformat=simple}
  \renewcommand\thesubfigure{(\alph{subfigure})}

% Put table captions on top / adjust the spacing
\captionsetup[table]{
    position=above,
    belowskip=4pt
  }
% \usepackage[skip=3.5\baselineskip]{caption}
% \captionsetup[table]{0.5\baselineskip}
% \captionsetup[table]{skip=90pt}
% \usepackage{float}
%   \floatstyle{plaintop}
%   \restylefloat{table}