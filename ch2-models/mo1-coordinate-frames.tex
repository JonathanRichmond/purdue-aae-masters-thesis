\section{Coordinate Frames}\label{sec:CoordinateFrames}
In this investigation, Cartesian coordinate frames are employed to represent three-\\dimensional
vector quantities. Coordinate frames may remain fixed in space or rotate about their origin.
Selection of coordinate frame depends on the specific application as it can be advantageous to
either position the origin at the center of mass of the system (barycenter) or align it with a
primary body of interest.

\subsection{The Barycentric Rotating Frame}
In a CR3BP system, the behavior of a spacecraft is best illustrated within a rotating frame with
the origin at the system barycenter. The $\xhat$-axis is defined to extend from the barycenter
toward the smaller primary body, while the $\zhat$-axis aligns with the system angular momentum
vector. Completing the triad, the $\yhat$-axis is evaluated as $\yhat=\zhat\times\xhat$. This set
of unit vectors rotates about the barycenter at a constant angular rate $n$ identical to the mean
motion of the two primaries. Consequently, the primary bodies remain fixed in place in the
rotating frame.

\subsection{The Arbitrary Barycentric Inertial Frame}
The rotating frame is also defined relative to the arbitrary barycentric inertial
(non-accelerating) frame by an angle $\theta$. When $\theta=0$ (arbitrarily defined to be true when
$t=0$ in the time-autonomous CR3BP), the two frames are aligned and as time progresses, $\theta$
increases as the rotating frame revolves around the shared origin with $\thetadot=n$. The arbitrary
inertial frame is denoted with the right-handed unit vectors $\Xhat$, $\Yhat$, and $\Zhat$, where
$\Zhat=\zhat$. In \cref{fig:baryFrames}, the barycentric $\{\xhat,\yhat,\zhat\}$ rotating frame and
$\{\Xhat,\Yhat,\Zhat\}$ arbitrary inertial frame for a sample CR3BP system are illustrated, with
their common origin centered at the barycenter of the primaries, $P_{1}$ and $P_{2}$.

\begin{figure}[H]
    \centering
    \includegraphics[width=0.5\textwidth]{figures/BaryFrames.jpg}
    \caption{Barycentric rotating and arbitrary inertial frames in a CR3BP system.}
    \label{fig:baryFrames}
\end{figure}

\subsection{The Ecliptic J2000 Primary-Centered Frame}
A commonly employed primary-centered reference frame is the Ecliptic J2000. As the name implies,
this frame is defined with its origin at the center of Earth and the Sun-Earth orbital (ecliptic)
plane on January 1, 2000 as the $\Xhat_{Ec}\Yhat_{Ec}$-plane. The $\Xhat_{Ec}$-axis is directed
towards the vernal equinox, the line of intersection between the Earth equatorial and ecliptic
planes on January 1, 2000. The $\Zhat_{Ec}$-axis is orthogonal to the ecliptic plane, and the
$\Yhat_{Ec}$-axis completes the triad, defined as $\Yhat_{Ec}=\Zhat_{Ec}\times\Xhat_{Ec}$. In this
investigation, the Ecliptic J2000 frame is utilized with its origin translated to the center of the
Sun, the shared primary body for the patched dynamical model. The Ecliptic J2000 coordinate frame,
as illustrated in \cref{fig:eclipJ2000Frame}, is computed from the Navigation and Ancillary
Information Facility (NAIF) SPICE ephemeris toolkit\cite{Semenov:2023}.

\begin{figure}[H]
    \centering
    \includegraphics[width=0.5\textwidth]{figures/EclipJ2000Frame.jpg}
    \caption{Earth-centered Ecliptic J2000 frame.}
    \label{fig:eclipJ2000Frame}
\end{figure}
