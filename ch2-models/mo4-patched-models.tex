\section{Patched Dynamical Models}\label{sec:PatchedModels}
A variety of methods exist to model the gravitational forces of three (or more) celestial bodies in
dynamical systems. While a high-fidelity ephemeris force model (HFEM) provides the most accuracy
among these multi-body models, some utilize simplifying assumptions to reduce computations and gain
more insight into the dynamics of the system while maintaining adequate fidelity. For including all
of the bodies in one model, there exist several 4-body problems that differ in layout, coherency,
and fidelity. Some of the more prominent options are the Bi-Circular Restriced 4-Body Problem
(BCR4BP)\cite{Boudad:2018}, Hills Restricted 4-Body Problem (HR4BP)\cite{Scheeres:1998}, and
Quasi-Bicircular Restricted 4-Body Problem (QBCR4BP)\cite{Andreu:2002}. A different approach, and
the one employed in this investigation, is to patch together 2BP and CR3BP models to build a larger
model to represent the dynamics. Two such patched models are outlined here.

\subsection{The Patched 2BP-CR3BP Model}
A patched 2BP-CR3BP model describes trajectories as they move between CR3BP systems through a 2BP
system. An example from this investigation is leaving the Sun-Earth CR3BP into heliocentric space
(modeled as a 2BP) before entering the Sun-Mars CR3BP. While the spacecraft is near a planet, it is
modeled in the Sun-planet CR3BP system. But once it reaches a specified distance from that planet,
it is modeled instead as Keplearian 2BP motion around the Sun\cite{Canales:2021b}. The interface
location between the two models is called the Sphere of Influence (SoI) since it represents where
the gravitational influence of the planet becomes negligible compared to that of the Sun. This
approach enables a seamless transition between dynamical systems, providing a robust framework for
analyzing interplanetary trajectories with high fidelity.

Trajectories computed in this patched model are best represented in a coordinate frame centered at
the focus of the 2BP, the Sun in this example. In this investigation, trajectories in the 2BP-CR3BP
patched model are viewed in the Sun-centered Ecliptic J2000 frame, as described in
\cref{sec:CoordinateFrames}. Although the Sun-Earth CR3BP is coplanar with the Ecliptic frame, the
Sun-Mars CR3BP system is not, so the Martian orbit is considered to be at its respective orbital
inclination relative to the Sun-Earth ecliptic plane. The $XY$-projection of an example 2BP-CR3BP
patched model system is provided in \cref{fig:2BP-CR3BP}. This representation ensures consistency
and clarity in visualizing trajectories across different dynamical systems, accommodating the truly
inclined orbital planes of celestial bodies.

\begin{figure}[H]
    \centering
    \includegraphics[width=0.75\textwidth]{figures/TBP-CR3BP.jpg}
    \caption{$XY$-Projection of the Patched 2BP-CR3BP Model}
    \label{fig:2BP-CR3BP}
\end{figure}

The radius of the SoI is a design parameter dependent on the application. For the patched model,
an SoI is desired such that many of the CR3BP periodic orbits around the Lagrange points are
included within the sphere, demonstrated in \cref{fig:SoI}. By defining a gravitational ratio:
\begin{equation}
    d_{SoI}=\frac{g_{2}}{g_{1}},
    \label{eq:patchedSoI}
\end{equation}
where $g_{i}$ is the gravitational acceleration of the respective primary body at a specified
location, an SoI radius from the planet is selected so that $d_{SoI}$ is sufficiently small,
i.e., the osculating (instantaneous) orbital elements remain near constant in the
CR3BP\cite{Canales:2021b}. This approach ensures that the SoI is appropriately sized to encompass
the relevant CR3BP dynamics near the planet.

\begin{figure}[H]
    \centering
    \includegraphics[width=0.9\textwidth]{figures/SoI.pdf}
    \caption{Patched 2BP-CR3BP sphere of influence around Earth, encompassing a large portion of the Sun-Earth $L_{2}$ Lyapunov family.}
    \label{fig:SoI}
\end{figure}

\subsection{The Blended CR3BP Model}
Two CR3BP models are also blended to form a 4-body model if one of the primary bodies is present in
both models. For example, a Sun-Earth CR3BP is blended with an Earth-Moon CR3BP to form a
Sun-Earth-Moon 4-body problem (here, the Earth is the common primary body). The blended model
incorporates the difference in inclinations between the two CR3BP models but is now a
time-dependent model\cite{Kakoi:2014}. Similar to the patched 2BP-CR3BP model, the boundary between
the two models is at an SoI, now around the smaller primary of the smaller CR3BP model (the Moon in
this example).

Unlike the patched model above, the blended model is best represented in the barycentric rotating
frame of the larger CR3BP model (the Sun-Earth rotating frame in this example). Since the blended
model is time-dependent, the portion of the trajectory computed in the smaller CR3BP is shifted
when represented in the larger model depending on the epoch. The $xy$-projection of an example
blended system is provided in \cref{fig:BlendedCR3BP}.

\begin{figure}[H]
    \centering
    \includegraphics[width=0.5\textwidth]{figures/BlendCR3BP.jpg}
    \caption{$xy$-Projection of the Blended CR3BP Model}
    \label{fig:BlendedCR3BP}
\end{figure}

The SoI radius employed for the blended model is different from that for the patched model. As
mentioned before, the SoI in this model is centered around the second primary of the smaller system
and the gravitational accelerations being compared are the first primary of the larger system and
the smaller primary of the second system (e.g., the Sun and the Moon). A blended CR3BP SoI radius
is defined as:
\begin{equation}
    r_{SoI}=l^{*}_{12}(\frac{m_{3}}{m_{1}})^{2/5},
    \label{eq:blendedSoI}
\end{equation}
where the primaries are numbered in order of decreasing mass\cite{Parker:2013}. This formulation
provides a SoI radius tailored to the blended model, ensuring an accurate representation of the
gravitational interactions across the hierarchical systems.
