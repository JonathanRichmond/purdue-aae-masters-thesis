\section{Coordinate Frame Transformations}
Since the patched and blended models used in this investigation use a variety of models centered
around different bodies, it is helpful to be able to view any trajectories in multiple reference
frames. As mentioned previously, reference frames can be inertial or rotating, and an important
component of interplanetary trajectory design is the ability to transform states and trajectories
between these two types of frames. A few representative example coordinate frame transformations
follow.

\subsection{Barycentric Rotating Frame - Primary-Centric Arbitrary Inertial Frame}
Most trajectories in a CR3BP are constructed in the barycentric rotating frame where
\cref{eq:EoMx}-\cref{eq:EoMz} are defined (see \cref{fig:baryFrames}). However, it can also be
beneficial to view these in an inertial frame centered on one of the primary gravitational bodies.
An arbitrary inertial frame can be defined where the inertial unit vectors $\{\Xhat,\Yhat,\Zhat\}$
are equivalent to the rotating unit vectors $\{\xhat,\yhat,\zhat\}$ at time $t_{0}$ (although the
frame centers may be in different locations).

The following steps will transform (nondimensionalized) states from the barycentric rotating frame
to a primary-centric arbitrary inertial frame:

\begin{enumerate}
    \item   Translate the position states from barycentric to primary-centric:
            \begin{equation}
                \rhobar_{P\rightarrow s/c}=\rhobar_{s/c}-\rhobar_{P}.
                \label{eq:translation}
            \end{equation}
    \item   Rotate the states depending on time since $t_{0}$
    
            Recall that the mean motion $n$ of the rotating frame is constant. When
            nondimensionalized in the CR3BP, $\ntilde=1$ and therefore the rtoation angle is just
            $\tau-\tau_{0}$. Since the $\zhat$- and $\Zhat$-axes coincide for an arbitrary inertial
            frame, a simple rotation matrix can be used to rotate the position states:
            \begin{equation}
                \Pbar=\begin{bmatrix}   \cos(\tau-\tau_{0}) &   -\sin(\tau-\tau_{0})    &   0   \\
                                        \sin(\tau-\tau_{0}) &   \cos(\tau-\tau_{0})     &   0   \\
                                        0                   &   0                       &   1   \end{bmatrix}\rhobar=\prescript{I}{}{C}^{R}\rhobar,
                \label{eq:positionrotation}
            \end{equation}
            where $\rhobar$ is the rotating position and $\Pbar$ is the inertial position.

            Basic kinematics can be used to compute the velocity in the rotating frame relative to
            an inertial observer:
            \begin{equation}
                \frac{\prescript{I}{}{d\rhobar}}{d\tau}=\frac{\prescript{R}{}{d\rhobar}}{d\tau}+\prescript{I}{}{\omegabar}^{R}\times\rhobar=\rhobardot+\zhat\times\rhobar,
                \label{eq:BKE}
            \end{equation}
            where $\prescript{I}{}{\omegabar}^{R}=\ntilde\zhat$ is the angular velocity relating
            the two frames. Therefore:
            \begin{equation}
                \frac{\prescript{I}{}{d\rhobar}}{d\tau}=(\xdot-y)\xhat+(\ydot+x)\yhat+\zdot\zhat.
                \label{eq:inertialrotatingvelocity}
            \end{equation}
            
            Using the rotation matrix $\prescript{I}{}{C}^{R}$ from \cref{eq:positionrotation},
            \cref{eq:inertialrotatingvelocity} can be written in matrix from and combined with the
            position rotation to achieve full state rotation:
            \begin{equation}
                \Qbar=\begin{bmatrix}   \prescript{I}{}{C}^{R}      &   \zerobar                \\
                                        \prescript{I}{}{\dot{C}}^{R} &   \prescript{I}{}{C}^{R}  \end{bmatrix}\qbar,
                \label{eq:rotation}
            \end{equation}
            where $\qbar$ is the rotating state and $\Qbar$ is the inertial state.
    \item   Dimensionalize the states if desired (see Section 2.3.2).
\end{enumerate}

To transform a primary-centric arbitrary inertial state to a barycentric rotating state, just
reverse the above states (nondimensionalizing if necessary) and invert the state rotation matrix.

\subsection{Barycentric Rotating Frame - Ecliptic J2000 Inertial Frame}
When desigining a trajectory across multiple systems, it is often useful to view each part of the
trajectory in a common inertial reference frame. In this investigation, the Earth Ecliptic J2000
inertial frame, introduced in Section 2.1.2 (\cref{fig:eclipJ2000Frame}), is used as the common
frame for interplanetary trajectories.

The transformation between barycentric rotating frame states and a primary-centric Ecliptic J2000
inertial frame states follows a similar process to the arbitrary inertial frame. However, since
this frame is defined by a particular epoch (January 1, 2000), the frame rotation is
epoch-dependent:

\begin{enumerate}
    \item   To properly compare the rotating frame to the Ecliptic J2000 inertial frame, the
            location of the second primary in its orbit at each epoch of the trajectory is needed.
            This is obtained by retrieving the orbital elements of the second primary at a selected
            initial epoch from SPICE\cite{Semenov:2023}. These orbital elements are then modified
            to match the CR3BP orbit assumptions ($a=l^{*}$ and $e=0$). Since the mean
            motion/angular velocity in the CR3Bp is constant at $\ntilde=1$:
            \begin{equation}
                \theta=\tau-\tau_{0}+\theta_{0},
                \label{eq:instantaneoustrueanomaly}
            \end{equation}
            where $\theta_{0}$ is the true anomaly at the initial epoch $t-{0}$ obtained from
            SPICE. These updated orbital elements are then used to calculate the full state vector
            (in dimensional units) of the second primary relative to the first using
            \cref{eq:eccentricanomaly}-\cref{eq:velocityvector}.
    \item   Dimensionalize the trajectory states, times, and angular velocity (see Section 2.3.2).
    \item   At each time, translate the position states from barycentric to primary-centric using
            \cref{eq:translation} (note that dimensional values should be used).
    \item   Define the instantaneous state rotation matrix using the second primary's Ecliptic
            J2000 state vector and angular momentum $\ambar$ (\cref{eq:angularmomentum}) at each
            time:
            \begin{equation}
                \xhat=\frac{\Rbar_{P_{1}\rightarrow P_{2}}}{|\Rbar_{P_{1}\rightarrow P_{2}}|},
                \label{eq:xhat}
            \end{equation}
            \begin{equation}
                \zhat=\frac{\ambar}{|\ambar|},
                \label{eq:zhat}
            \end{equation}
            \begin{equation}
                \yhat=\zhat\times\xhat,
                \label{eq:yhat}
            \end{equation}
            \begin{equation}
                \prescript{Ec}{}{C}^{R}=\begin{bmatrix} \xhat   &   \yhat   &   \zhat   \end{bmatrix}.
                \label{eq:eclipticpositionrotation}
            \end{equation}

            The full state rotation matrix can be found through the same process used in Section
            2.5.1, using a dimensional angular velocity:
            \begin{equation}
                \prescript{Ec}{}{\dot{C}}^{R}=\begin{bmatrix}   n\yhat  &   -n\xhat &   \zerobar    \end{bmatrix}.
                \label{eq:eclipticvelocityrotation}
            \end{equation}
            in \cref{eq:rotation} with dimensional values.
    \item   Nondimensionalize the states if desired.
\end{enumerate}

States can be transformed from a primary-centric Ecliptic J2000 inertial frame to a barycentric
rotating frame by reversing the above steps and inverting the state rotation matrix. This
becomes a useful tool when designing interplanetary trajectories using multi-body dynamics.
