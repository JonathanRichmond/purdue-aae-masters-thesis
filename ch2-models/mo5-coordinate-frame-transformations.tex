\section{Coordinate Frame Transformations}\label{sec:FrameTransformations}
Since the patched and blended models in this investigation employ a variety of models centered
around different bodies, it is helpful to be able to view any trajectories in multiple reference
frames. As mentioned previously, reference frames can be rotating or stationary, and an important
component of interplanetary trajectory design is the ability to transform states and trajectories
between the two types of frames. A few representative examples of coordinate frame transformations
employed in this investigation follow.

\subsection{Barycentric Rotating Frame - Primary-Centered Arbitrary Inertial Frame}
Most trajectories in a CR3BP are constructed in the barycentric rotating frame where
\cref{eq:EoMx}-\cref{eq:EoMz} are defined (see \cref{fig:baryFrames}). However, it can also be
beneficial to view trajectories in an inertial reference frame. An arbitrary inertial frame is
defined where the inertial unit vectors $\{\Xhat,\Yhat,\Zhat\}$ are equivalent to the rotating unit
vectors $\{\xhat,\yhat,\zhat\}$ at time $t_{0}$. The origin of the unit vectors is also translated
to the center of the primary body.

The following steps transform (nondimensionalized) states from the barycentric rotating frame to
the primary-centered arbitrary frame:
\begin{enumerate}
    \item   Translate the position states from barycentric to primary-centric:
            \begin{equation}
                \rhobar_{P\rightarrow s/c}=\rhobar_{s/c}-\rhobar_{P}.
                \label{eq:translation}
            \end{equation}
    \item   Rotate the states depending on time since $t_{0}$. Recall that the mean motion $n$ of
            the rotating frame is constant. When nondimensionalized in the CR3BP, $\ntilde=1$ and
            consequently the rotation angle are just $\tau-\tau_{0}$. Since the $\zhat$- and
            $\Zhat$-axes coincide for an arbitrary inertial frame, a simple rotation matrix is
            utilized to rotate the position states:
            \begin{equation}
                \Pbar=\begin{bmatrix}   \cos(\tau-\tau_{0}) &   -\sin(\tau-\tau_{0})    &   0   \\
                                        \sin(\tau-\tau_{0}) &   \cos(\tau-\tau_{0})     &   0   \\
                                        0                   &   0                       &   1   \end{bmatrix}\rhobar=\prescript{I}{}{C}^{R}\rhobar,
                \label{eq:positionrotation}
            \end{equation}
            where $\rhobar$ is the rotating position and $\Pbar$ is the inertial position. The
            basic kinematic equation is then employed to compute the velocity in the rotating frame
            relative to an inertial observer:
            \begin{equation}
                \frac{\prescript{I}{}{d\rhobar}}{d\tau}=\frac{\prescript{R}{}{d\rhobar}}{d\tau}+\prescript{I}{}{\omegabar}^{R}\times\rhobar=\rhobardot+\zhat\times\rhobar,
                \label{eq:BKE}
            \end{equation}
            where $\prescript{I}{}{\omegabar}^{R}=\ntilde\zhat$ is the angular velocity relating
            the two frames. Consequently:
            \begin{equation}
                \frac{\prescript{I}{}{d\rhobar}}{d\tau}=(\xdot-y)\xhat+(\ydot+x)\yhat+\zdot\zhat.
                \label{eq:inertialrotatingvelocity}
            \end{equation}
            Utilizing the rotation matrix $\prescript{I}{}{C}^{R}$ from \cref{eq:positionrotation},
            \cref{eq:inertialrotatingvelocity} is expressed in matrix form and combined with the
            position rotation to achieve full state rotation:
            \begin{equation}
                \Qbar=\begin{bmatrix}   \prescript{I}{}{C}^{R}      &   \zerobar                \\
                                        \prescript{I}{}{\dot{C}}^{R} &   \prescript{I}{}{C}^{R}  \end{bmatrix}\qbar,
                \label{eq:rotation}
            \end{equation}
            where $\qbar$ is the rotating state and $\Qbar$ is the inertial state.
    \item   Dimensionalize the states if desired (see \cref{sec:CR3BP}).
\end{enumerate}
To transform a primary-centric arbitrary inertial state to a barycentric rotating state, just
reverse the above process (nondimensionalizing the states if necessary), inverting the state
rotation matrix.

\subsection{Barycentric Rotating Frame - Ecliptic J2000 Frame}
When designing a trajectory across multiple systems, it is often beneficial to view each part of
the trajectory in a common reference frame. In this investigation, the Earth Ecliptic J2000 frame,
introduced in \cref{sec:CoordinateFrames} (\cref{fig:eclipJ2000Frame}), is employed as the common
frame for viewing interplanetary trajectories.

The transformation between barycentric rotating frame states and primary-centric Ecliptic J2000
frame states follows a similar process to that for the arbitrary inertial frame. However, since
the Ecliptic J2000 frame is defined by a particular epoch (January 1, 2000), the frame rotation is
epoch-dependent:
\begin{enumerate}
    \item   Determine the state vector of the second primary with respect to the first in the
            Ecliptic J2000 frame. To properly compare the two frames, the location of the second
            primary in its orbit at each epoch of the trajectory is required. This is obtained by
            retrieving the orbital elements of the second primary at a selected initial epoch from
            SPICE kernels\cite{Semenov:2023}. The orbital elements are then modified to match the
            CR3BP orbit assumptions ($a=l^{*}$ and $e=0$). Since the mean motion/angular velocity
            in the CR3BP is constant at $\ntilde=1$:
            \begin{equation}
                \theta=\tau-\tau_{0}+\theta_{0},
                \label{eq:instantaneoustrueanomaly}
            \end{equation}
            where $\theta_{0}$ is the true anomaly at the initial epoch $t_{0}$ obtained from
            the SPICE kernels. The updated orbital elements are then employed to calculate the full
            state vector (in dimensional units) of the second primary relative to the first with
            \cref{eq:eccentricanomaly}-\cref{eq:velocityvector}.
    \item   Dimensionalize the trajectory states, times, and angular velocity (see
            \cref{sec:CR3BP}).
    \item   At each time, translate the position states from barycentric to primary-centric with
            \cref{eq:translation} (note that the values should be in dimensional units).
    \item   Define the instantaneous state rotation matrix utilizing the Ecliptic J2000 state
            vector of the second primary and angular momentum $\ambar$ (\cref{eq:angularmomentum})
            at each time:
            \begin{equation}
                \xhat=\frac{\Rbar_{P_{1}\rightarrow P_{2}}}{|\Rbar_{P_{1}\rightarrow P_{2}}|},
                \label{eq:xhat}
            \end{equation}
            \vspace{1mm}
            \begin{equation}
                \zhat=\frac{\ambar}{|\ambar|},
                \label{eq:zhat}
            \end{equation}
            \vspace{1mm}
            \begin{equation}
                \yhat=\zhat\times\xhat,
                \label{eq:yhat}
            \end{equation}
            \vspace{1mm}
            \begin{equation}
                \prescript{Ec}{}{C}^{R}=\begin{bmatrix} \xhat   &   \yhat   &   \zhat   \end{bmatrix}.
                \label{eq:eclipticpositionrotation}
            \end{equation}
            The full state rotation matrix is found through the same process as before, utilizing a
            dimensional angular velocity:
            \begin{equation}
                \prescript{Ec}{}{\dot{C}}^{R}=\begin{bmatrix}   n\yhat  &   -n\xhat &   \zerobar    \end{bmatrix}.
                \label{eq:eclipticvelocityrotation}
            \end{equation}
            in \cref{eq:rotation} with dimensional values.
    \item   Nondimensionalize the states if desired.
\end{enumerate}
States are transformed from a primary-centric Ecliptic J2000 frame to a barycentric rotating frame
by reversing the above steps, inverting the state rotation matrix.
