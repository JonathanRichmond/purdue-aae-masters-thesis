\section{The Circular Restricted Three-Body Problem}
When a spacecraft is significantly impacted by the gravitational force of
two celestial bodies the Circular Restricted 3-Body Problem better approximates the spacecraft's
motion compared to two-body problems. Therefore, this investigation uses the CR3BP to model the
Earth-Moon and Sun-planet systems when appropriate. The CR3BP is an autonomous model (its dynamics
are not time-dependent) that provides insight into the dynamical structures present in the system
without some of the complexities of a higher-fidelity ephemeris model.

\subsection{Equations of Motion}
The CR3BP consisits of three primary bodies, two celestial bodies and a masless spacecraft. The two
celestial bodies exert gravitational forces on each other and the satellite; however, the satellite
does not affect the other two bodies. 

The two celestial bodies are treated as point masses and assumed to move in circular orbits, with a
constant angular velocity, around their barycenter $B$. Assuming that no other forces are acting on
the system, $B$ can be considered an inertial point and similar to the 2BP, Newton's Laws can be
expressed relative to that point. Unlike the 2BP, there is currently no analytical solution to
represent the dynamics of the CR3BP. Consequently, all trajectories in this model must be
numerically propagated in time using nonlinear, coupled equations of motion.

It is also useful and common practice to represent these equations and visualize them in a
barycentric rotating coordinate frame, $\{\xhat,\yhat,\zhat\}$, as shown by the dashed lines in
\cref{fig:baryFrames} and described in Section 2.1. In this frame, the two celestial primaries
remain fixed, while the spacecraft moves relative to them in three-dimensional configuration space.

A single mass ratio $\mu$ characterizes a CR3BP system:
\begin{equation}
    \mu=\frac{m_{2}}{m_{1}+m_{2}},
    \label{eq:mu}
\end{equation}
where $m_{1}$ and $m_{2}$ are the masses of the larger and smaller celestial primaries,
respectively. Using this parameter, a pseudo-potential function $U$ describes the gravitational
forces on the system expressed in the barycentric rotating frame:
\begin{equation}
    U=\frac{1}{2}(x^{2}+y^{2})+\frac{1-\mu}{d}+\frac{\mu}{r},
    \label{eq:pseud0-potential}
\end{equation}
\begin{equation}
    d=\sqrt{(x+\mu)^{2}+y^{2}+z^{2}},
    \label{eq:P1distance}
\end{equation}
\begin{equation}
    r=\sqrt{(x-1+\mu)^{2}+y^{2}+z^{2}},
    \label{eq:P2distance}
\end{equation}
where here, $d$ and $r$ are the distances from $P_{1}$ and $P_{2}$, respectively. From the pseudo-
potential, the scalar nonlinear equations of motion are expressed in the barycentric rotating
frame:
\begin{equation}
    \xddot=2\ydot+\frac{\partial U}{\partial x},
    \label{eq:EoMx}
\end{equation}
\begin{equation}
    \yddot=-2\xdot+\frac{\partial U}{\partial y},
    \label{eq:EoMy}
\end{equation}
\begin{equation}
    \zddot=\frac{\partial U}{\partial z}.
    \label{eq:EoMz}
\end{equation}.

\subsection{Nondimensionalized Values}
