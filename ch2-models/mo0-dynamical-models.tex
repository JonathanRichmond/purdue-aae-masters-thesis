\chapter{DYNAMICAL MODELS}

In the cislunar and other multi-body environments, the gravitational accelerations from one or more
masses can affect spacecraft trajectories simultaneously, depending on their proximity to the
vehicle. Multiple dynamical models are available to describe the spacecraft behavior; however, as
the fidelity of the models increases, so does their complexity. Consequently, the appropriate
dynamical model for a given application is a balance between accuracy and simplicity. This
investigation is based on two primary dynamical models: The relative 2-Body Problem (2BP) and the
Circular Restricted 3-Body Problem (CR3BP). The 2BP serves as a model for spacecraft dynamics when
its behavior is well-represented by assuming it is solely governed by the gravitational influence
of a single body, primarily applied to heliocentric arcs along a trajectory. In cases where the
dynamics are significantly influenced by the gravitational forces of two bodies, for example, in
Sun-planet or the Earth-Moon systems, the CR3BP offers a more accurate description of the
spacecraft behavior while remaining simple enough to provide insight into the general dynamics.

\section{Coordinate Frames}\label{sec:CoordinateFrames}
In this investigation, Cartesian coordinate frames are employed to represent three-\\dimensional
vector quantities. Coordinate frames may remain fixed in space or rotate about their origin.
Selection of coordinate frame depends on the specific application as it can be advantageous to
either position the origin at the center of mass of the system (barycenter) or align it with a
primary body of interest.

\subsection{The Barycentric Rotating Frame}
In a CR3BP system, the behavior of a spacecraft is best illustrated within a rotating frame with
the origin at the system barycenter. The $\xhat$-axis is defined to extend from the barycenter
toward the smaller primary body, while the $\zhat$-axis aligns with the system angular momentum
vector. Completing the triad, the $\yhat$-axis is evaluated as $\yhat=\zhat\times\xhat$. This set
of unit vectors rotates about the barycenter at a constant angular rate $n$ identical to the mean
motion of the two primaries. Consequently, the primary bodies remain fixed in place in the
rotating frame.

\subsection{The Arbitrary Barycentric Inertial Frame}
The rotating frame is also defined relative to the arbitrary barycentric inertial
(non-accelerating) frame by an angle $\theta$. When $\theta=0$ (arbitrarily defined to be true when
$t=0$ in the time-autonomous CR3BP), the two frames are aligned and as time progresses, $\theta$
increases as the rotating frame revolves around the shared origin with $\thetadot=n$. The arbitrary
inertial frame is denoted with the right-handed unit vectors $\Xhat$, $\Yhat$, and $\Zhat$, where
$\Zhat=\zhat$. In \cref{fig:baryFrames}, the barycentric $\{\xhat,\yhat,\zhat\}$ rotating frame and
$\{\Xhat,\Yhat,\Zhat\}$ arbitrary inertial frame for a sample CR3BP system are illustrated, with
their common origin centered at the barycenter of the primaries, $P_{1}$ and $P_{2}$.

\begin{figure}[H]
    \centering
    \includegraphics[width=0.5\textwidth]{figures/BaryFrames.jpg}
    \caption{Barycentric rotating and arbitrary inertial frames in a CR3BP system.}
    \label{fig:baryFrames}
\end{figure}

\subsection{The Ecliptic J2000 Primary-Centered Frame}
A commonly employed primary-centered reference frame is the Ecliptic J2000. As the name implies,
this frame is defined with its origin at the center of Earth and the Sun-Earth orbital (ecliptic)
plane on January 1, 2000 as the $\Xhat_{Ec}\Yhat_{Ec}$-plane. The $\Xhat_{Ec}$-axis is directed
towards the vernal equinox, the line of intersection between the Earth equatorial and ecliptic
planes on January 1, 2000. The $\Zhat_{Ec}$-axis is orthogonal to the ecliptic plane, and the
$\Yhat_{Ec}$-axis completes the triad, defined as $\Yhat_{Ec}=\Zhat_{Ec}\times\Xhat_{Ec}$. In this
investigation, the Ecliptic J2000 frame is utilized with its origin translated to the center of the
Sun, the shared primary body for the patched dynamical model. The Ecliptic J2000 coordinate frame,
as illustrated in \cref{fig:eclipJ2000Frame}, is computed from the Navigation and Ancillary
Information Facility (NAIF) SPICE ephemeris toolkit\cite{Semenov:2023}.

\begin{figure}[H]
    \centering
    \includegraphics[width=0.5\textwidth]{figures/EclipJ2000Frame.jpg}
    \caption{Earth-centered Ecliptic J2000 frame.}
    \label{fig:eclipJ2000Frame}
\end{figure}

\section{The Two-Body Problem}
\section{The Circular Restricted Three-Body Problem}

\section{Patched Dynamical Models}\label{sec:PatchedModels}
A variety of methods exist to model the gravitational forces of three (or more) celestial bodies in
dynamical systems. While a high-fidelity ephemeris force model (HFEM) provides the most accuracy
among these multi-body models, some utilize simplifying assumptions to reduce computations and gain
more insight into the dynamics of the system while maintaining adequate fidelity. For including all
of the bodies in one model, there exist several 4-body problems that differ in layout, coherency,
and fidelity. Some of the more prominent options are the Bi-Circular Restriced 4-Body Problem
(BCR4BP)\cite{Boudad:2018}, Hills Restricted 4-Body Problem (HR4BP)\cite{Scheeres:1998}, and
Quasi-Bicircular Restricted 4-Body Problem (QBCR4BP)\cite{Andreu:2002}. A different approach, and
the one employed in this investigation, is to patch together 2BP and CR3BP models to build a larger
model to represent the dynamics. Two such patched models are outlined here.

\subsection{The Patched 2BP-CR3BP Model}
A patched 2BP-CR3BP model describes trajectories as they move between CR3BP systems through a 2BP
system. An example from this investigation is leaving the Sun-Earth CR3BP into heliocentric space
(modeled as a 2BP) before entering the Sun-Mars CR3BP. While the spacecraft is near a planet, it is
modeled in the Sun-planet CR3BP system. But once it reaches a specified distance from that planet,
it is modeled instead as Keplearian 2BP motion around the Sun\cite{Canales:2021b}. The interface
location between the two models is called the Sphere of Influence (SoI) since it represents where
the gravitational influence of the planet becomes negligible compared to that of the Sun. This
approach enables a seamless transition between dynamical systems, providing a robust framework for
analyzing interplanetary trajectories with high fidelity.

Trajectories computed in this patched model are best represented in a coordinate frame centered at
the focus of the 2BP, the Sun in this example. In this investigation, trajectories in the 2BP-CR3BP
patched model are viewed in the Sun-centered Ecliptic J2000 frame, as described in
\cref{sec:CoordinateFrames}. Although the Sun-Earth CR3BP is coplanar with the Ecliptic frame, the
Sun-Mars CR3BP system is not, so the Martian orbit is considered to be at its respective orbital
inclination relative to the Sun-Earth ecliptic plane. The $XY$-projection of an example 2BP-CR3BP
patched model system is provided in \cref{fig:2BP-CR3BP}. This representation ensures consistency
and clarity in visualizing trajectories across different dynamical systems, accommodating the truly
inclined orbital planes of celestial bodies.

\begin{figure}[H]
    \centering
    \includegraphics[width=0.75\textwidth]{figures/TBP-CR3BP.jpg}
    \caption{$XY$-Projection of the Patched 2BP-CR3BP Model}
    \label{fig:2BP-CR3BP}
\end{figure}

The radius of the SoI is a design parameter dependent on the application. For the patched model,
an SoI is desired such that many of the CR3BP periodic orbits around the Lagrange points are
included within the sphere, demonstrated in \cref{fig:SoI}. By defining a gravitational ratio:
\begin{equation}
    d_{SoI}=\frac{g_{2}}{g_{1}},
    \label{eq:patchedSoI}
\end{equation}
where $g_{i}$ is the gravitational acceleration of the respective primary body at a specified
location, an SoI radius from the planet is selected so that $d_{SoI}$ is sufficiently small,
i.e., the osculating (instantaneous) orbital elements remain near constant in the
CR3BP\cite{Canales:2021b}. This approach ensures that the SoI is appropriately sized to encompass
the relevant CR3BP dynamics near the planet.

\begin{figure}[H]
    \centering
    \includegraphics[width=0.9\textwidth]{figures/SoI.pdf}
    \caption{Patched 2BP-CR3BP sphere of influence around Earth, encompassing a large portion of the Sun-Earth $L_{2}$ Lyapunov family.}
    \label{fig:SoI}
\end{figure}

\subsection{The Blended CR3BP Model}
Two CR3BP models are also blended to form a 4-body model if one of the primary bodies is present in
both models. For example, a Sun-Earth CR3BP is blended with an Earth-Moon CR3BP to form a
Sun-Earth-Moon 4-body problem (here, the Earth is the common primary body). The blended model
incorporates the difference in inclinations between the two CR3BP models but is now a
time-dependent model\cite{Kakoi:2014}. Similar to the patched 2BP-CR3BP model, the boundary between
the two models is at an SoI, now around the smaller primary of the smaller CR3BP model (the Moon in
this example).

Unlike the patched model above, the blended model is best represented in the barycentric rotating
frame of the larger CR3BP model (the Sun-Earth rotating frame in this example). Since the blended
model is time-dependent, the portion of the trajectory computed in the smaller CR3BP is shifted
when represented in the larger model depending on the epoch. The $xy$-projection of an example
blended system is provided in \cref{fig:BlendedCR3BP}.

\begin{figure}[H]
    \centering
    \includegraphics[width=0.5\textwidth]{figures/BlendCR3BP.jpg}
    \caption{$xy$-Projection of the Blended CR3BP Model}
    \label{fig:BlendedCR3BP}
\end{figure}

The SoI radius employed for the blended model is different from that for the patched model. As
mentioned before, the SoI in this model is centered around the second primary of the smaller system
and the gravitational accelerations being compared are the first primary of the larger system and
the smaller primary of the second system (e.g., the Sun and the Moon). A blended CR3BP SoI radius
is defined as:
\begin{equation}
    r_{SoI}=l^{*}_{12}(\frac{m_{3}}{m_{1}})^{2/5},
    \label{eq:blendedSoI}
\end{equation}
where the primaries are numbered in order of decreasing mass\cite{Parker:2013}. This formulation
provides a SoI radius tailored to the blended model, ensuring an accurate representation of the
gravitational interactions across the hierarchical systems.

\section{Coordinate Frame Transformations}
Since the patched and blended models used in this investigation use a variety of models centered
around different bodies, it is helpful to be able to view any trajectories in multiple reference
frames. As mentioned previously, reference frames can be inertial or rotating, and an important
component of interplanetary trajectory design is the ability to transform states and trajectories
between these two types of frames. A few representative example coordinate frame transformations
follow.

\subsection{Barycentric Rotating Frame - Primary-Centric Arbitrary Inertial Frame}
Most trajectories in a CR3BP are constructed in the barycentric rotating frame where
\cref{eq:EoMx}-\cref{eq:EoMz} are defined (see \cref{fig:baryFrames}). However, it can also be
beneficial to view these in an inertial frame centered on one of the primary gravitational bodies.
An arbitrary inertial frame can be defined where the inertial unit vectors $\{\Xhat,\Yhat,\Zhat\}$
are equivalent to the rotating unit vectors $\{\xhat,\yhat,\zhat\}$ at time $t_{0}$ (although the
frame centers may be in different locations).

The following steps will transform (nondimensionalized) states from the barycentric rotating frame
to a primary-centric arbitrary inertial frame:

\begin{enumerate}
    \item   Translate the position states from barycentric to primary-centric:
            \begin{equation}
                \rhobar_{P\rightarrow s/c}=\rhobar_{s/c}-\rhobar_{P}.
                \label{eq:translation}
            \end{equation}
    \item   Rotate the states depending on time since $t_{0}$
    
            Recall that the mean motion $n$ of the rotating frame is constant. When
            nondimensionalized in the CR3BP, $\ntilde=1$ and therefore the rtoation angle is just
            $\tau-\tau_{0}$. Since the $\zhat$- and $\Zhat$-axes coincide for an arbitrary inertial
            frame, a simple rotation matrix can be used to rotate the position states:
            \begin{equation}
                \Pbar=\begin{bmatrix}   \cos(\tau-\tau_{0}) &   -\sin(\tau-\tau_{0})    &   0   \\
                                        \sin(\tau-\tau_{0}) &   \cos(\tau-\tau_{0})     &   0   \\
                                        0                   &   0                       &   1   \end{bmatrix}\rhobar=\prescript{I}{}{C}^{R}\rhobar,
                \label{eq:positionrotation}
            \end{equation}
            where $\rhobar$ is the rotating position and $\Pbar$ is the inertial position.

            Basic kinematics can be used to compute the velocity in the rotating frame relative to
            an inertial observer:
            \begin{equation}
                \frac{\prescript{I}{}{d\rhobar}}{d\tau}=\frac{\prescript{R}{}{d\rhobar}}{d\tau}+\prescript{I}{}{\omegabar}^{R}\times\rhobar=\rhobardot+\zhat\times\rhobar,
                \label{eq:BKE}
            \end{equation}
            where $\prescript{I}{}{\omegabar}^{R}=\ntilde\zhat$ is the angular velocity relating
            the two frames. Therefore:
            \begin{equation}
                \frac{\prescript{I}{}{d\rhobar}}{d\tau}=(\xdot-y)\xhat+(\ydot+x)\yhat+\zdot\zhat.
                \label{eq:inertialrotatingvelocity}
            \end{equation}
            
            Using the rotation matrix $\prescript{I}{}{C}^{R}$ from \cref{eq:positionrotation},
            \cref{eq:inertialrotatingvelocity} can be written in matrix from and combined with the
            position rotation to achieve full state rotation:
            \begin{equation}
                \Qbar=\begin{bmatrix}   \prescript{I}{}{C}^{R}      &   \zerobar                \\
                                        \prescript{I}{}{\dot{C}}^{R} &   \prescript{I}{}{C}^{R}  \end{bmatrix}\qbar,
                \label{eq:rotation}
            \end{equation}
            where $\qbar$ is the rotating state and $\Qbar$ is the inertial state.
    \item   Dimensionalize the states if desired (see Section 2.3.2).
\end{enumerate}

To transform a primary-centric arbitrary inertial state to a barycentric rotating state, just
reverse the above states (nondimensionalizing if necessary) and invert the state rotation matrix.

\subsection{Barycentric Rotating Frame - Ecliptic J2000 Inertial Frame}
When desigining a trajectory across multiple systems, it is often useful to view each part of the
trajectory in a common inertial reference frame. In this investigation, the Earth Ecliptic J2000
inertial frame, introduced in Section 2.1.2 (\cref{fig:eclipJ2000Frame}), is used as the common
frame for interplanetary trajectories.

The transformation between barycentric rotating frame states and a primary-centric Ecliptic J2000
inertial frame states follows a similar process to the arbitrary inertial frame. However, since
this frame is defined by a particular epoch (January 1, 2000), the frame rotation is
epoch-dependent:

\begin{enumerate}
    \item   To properly compare the rotating frame to the Ecliptic J2000 inertial frame, the
            location of the second primary in its orbit at each epoch of the trajectory is needed.
            This is obtained by retrieving the orbital elements of the second primary at a selected
            initial epoch from SPICE\cite{Semenov:2023}. These orbital elements are then modified
            to match the CR3BP orbit assumptions ($a=l^{*}$ and $e=0$). Since the mean
            motion/angular velocity in the CR3Bp is constant at $\ntilde=1$:
            \begin{equation}
                \theta=\tau-\tau_{0}+\theta_{0},
                \label{eq:instantaneoustrueanomaly}
            \end{equation}
            where $\theta_{0}$ is the true anomaly at the initial epoch $t-{0}$ obtained from
            SPICE. These updated orbital elements are then used to calculate the full state vector
            (in dimensional units) of the second primary relative to the first using
            \cref{eq:eccentricanomaly}-\cref{eq:velocityvector}.
    \item   Dimensionalize the trajectory states, times, and angular velocity (see Section 2.3.2).
    \item   At each time, translate the position states from barycentric to primary-centric using
            \cref{eq:translation} (note that dimensional values should be used).
    \item   Define the instantaneous state rotation matrix using the second primary's Ecliptic
            J2000 state vector and angular momentum $\ambar$ (\cref{eq:angularmomentum}) at each
            time:
            \begin{equation}
                \xhat=\frac{\Rbar_{P_{1}\rightarrow P_{2}}}{|\Rbar_{P_{1}\rightarrow P_{2}}|},
                \label{eq:xhat}
            \end{equation}
            \begin{equation}
                \zhat=\frac{\ambar}{|\ambar|},
                \label{eq:zhat}
            \end{equation}
            \begin{equation}
                \yhat=\zhat\times\xhat,
                \label{eq:yhat}
            \end{equation}
            \begin{equation}
                \prescript{Ec}{}{C}^{R}=\begin{bmatrix} \xhat   &   \yhat   &   \zhat   \end{bmatrix}.
                \label{eq:eclipticpositionrotation}
            \end{equation}

            The full state rotation matrix can be found through the same process used in Section
            2.5.1, using a dimensional angular velocity:
            \begin{equation}
                \prescript{Ec}{}{\dot{C}}^{R}=\begin{bmatrix}   n\yhat  &   -n\xhat &   \zerobar    \end{bmatrix}.
                \label{eq:eclipticvelocityrotation}
            \end{equation}
            in \cref{eq:rotation} with dimensional values.
    \item   Nondimensionalize the states if desired.
\end{enumerate}

States can be transformed from a primary-centric Ecliptic J2000 inertial frame to a barycentric
rotating frame by reversing the above steps and inverting the state rotation matrix. This
becomes a useful tool when designing interplanetary trajectories using multi-body dynamics.

