\section{Comparison to Previous Work}\label{sec:PreviousWork}
As mentioned previously, several authors have investigated transfers between Earth or cislunar
space and Mars. The studies employ a variety of methodologies to accomplish their transfers,
leveraging direct impulsive burns, low-thrust, and/or invariant manifolds. Some of their relevant
findings are provided here for comparison to the results of this investigation. It is important to
note that each of the transfers from these investigations, whose results are consolidated into
\cref{tab:transferCosts}, utilizes a different methodology. Some employ an optimizer, others
leverage invariant manifolds, and the number and placement of maneuvers vary from transfer to
transfer. Additionally, each one departs from and arrives at different locations, making it
difficult to directly compare the total maneuver cost and TOF values. Many of the investigations
also assume that the orbital planes of Earth and Mars are coplanar, neglecting the nontrivial plane
change maneuver required when considering true orbital planes. Consequently, a direct comparison
cannot be made between all of the methodologies. However, they do provide some intuition for the
costs of similar interplanetary transfer types.

Kakoi et al. inspires the transfer methodology between the Earth-Moon and Sun-Earth CR3BP
models in this investigation\cite{Kakoi:2014,Kakoi:2015}. Their investigation focused on comparing
different transfer scenarios departing from Earth-Moon $L_{1}$ and $L_{2}$ halo orbits. While the
strategies employed are slightly different, Kakoi et al. construct transfers that utilize Sun-Earth
unstable invariant manifold arcs and compare them to direct transfers from Earth-Moon halo orbits.
As some sample values for an $L_{2}$ halo departure, their investigation found that a transfer
along a Sun-Earth halo unstable manifold arc required a total maneuver cost of $1.604$ km/s and a
total TOF of $1762$ days ($4.82$ years). Note that in their investigation, a maneuver is applied
along the departure manifold arc to speed up the transfer and the final destination is the
ephemeris location of Mars, cutting off the long arrival time of the stable manifold arc. On the
other hand, directly utilizing an unstable manifold arc from the Earth-Moon halo orbit resulted in
a transfer with a $\Delta v$ of $0.921$ km/s and a TOF of $1784$ days ($4.88$
years)\cite{Kakoi:2015}. The direct transfer again employs a maneuver along the departure manifold
arc, as well as an inclination change maneuver during an Earth-flyby. Other direct transfers
provided in the study have lower and higher times-of-flight and maneuver costs, demonstrating that
there are a wide variety of available transfer options.

\begin{table}[H]
    \centering
    \caption{Representative transfer costs from previous and current investigations.}
    \begin{tabular}{|c|c|c|c|c|}
        \hline
        \textbf{Transfer Type}                      &   \textbf{Origin} &   \textbf{Destination}    &   \boldmath$\Delta v$ \textbf{[km/s]} &   \boldmath$TOF$ \textbf{[days]}  \\  \hline
        Hohmann Transfer                            &   Earth           &   Sun-Mars $L_{1}$        &   5.64                                &   258                             \\  \hline
        Lambert Arc\cite{Eagle:2022}                &   Earth           &   Mars                    &   5.61                                &   310                             \\  \hline
        Optimized 4-Body\cite{Miele:1999}           &   LEO             &   LMO                     &   5.65                                &   258                             \\  \hline
        Lambert Arc CR3BP\cite{Conte:2017}          &   Lunar DRO       &   LMO                     &   3.29                                &   206                             \\  \hline
        CR3BP Manifolds\cite{Topputo:2005}          &   S-E Lyapunov    &   S-M Lyapunov            &   3.76                                &   999                             \\  \hline
        Pseudo-Manifolds\cite{Haibin:2014}          &   S-E Halo        &   S-M Halo                &   2.06                                &   596                             \\  \hline
        CR3BP Manifolds\cite{Kakoi:2014}            &   E-M Halo        &   Mars                    &   0.92                                &   1784                            \\  \hline
        CR3BP Manifolds\cite{Cavallari:2019}        &   Lunar DRO       &   S-M Lyapunov            &   5.40                                &   1102                            \\  \hline
        CR3BP MMAT\cite{Canales:2022}               &   S-E Halo        &   S-M Halo                &   4.53                                &   1776                            \\  \hline
        CR3BP MMAT                                  &   E-M Halo        &   S-M Halo                &   5.10                                &   1520                            \\  \hline
    \end{tabular}
    \label{tab:transferCosts}
\end{table}

The results of the current investigation are consistent with Kakoi's conclusions. Utilizing
invariant manifold arcs decreases transfer maneuver $\Delta v$ values while increasing the
TOF\cite{Kakoi:2015}, compared here to the results of previous investigations and direct comparison
to the Keplerian modified Hohmann transfer solution. While Kakoi's transfer data cannot be directly
compared to those from this investigation, the general conclusion of Kakoi et al. is that the
utilization of direct Earth-Moon manifolds allows for more flexible departure dates and transfer
options, while the comparison of transfers with direct departures to those with Sun-Earth staging
orbits is orbit-dependent\cite{Kakoi:2014,Kakoi:2015}. This investigation supports the flexibility
of the direct transfer methodology and concludes that the maneuver cost comparison is
orbit-dependent. However, employing the MMAT strategy to connect the planetary system invariant
manifold arcs, it appears that the options with direct departures have shorter times-of-flight
compared to when an intermediate staging orbit is included. This investigation reinforces Kakoi's
findings, highlighting the flexibility and reduced times-of-flight of direct transfers, while also
confirming that maneuver costs remain highly orbit-dependent.

From a direct comparison between the $\Delta v$ and TOF results of this investigation and the
previous studies summarized in \cref{tab:transferCosts}, incorporating invariant manifold arcs
decreases maneuver cost on the order of $1$ km/s compared to a modified Hohmann transfer, Lambert
arcs\cite{Eagle:2022}, and optimized direct solutions between LEO and LMO\cite{Miele:1999}. As
explained, the manifold arcs do significantly increase the transfer TOF, an acceptable trade-off
for some mission scenarios, depending on the transfer design. The results are also consistent with
some previous manifold arc transfers between CR3BP orbits\cite{Cavallari:2019,Canales:2022}.

At first glance, several of the entries in \cref{tab:transferCosts} that utilize manifold arcs
appear to provide better results than this investigation. However, the transfers cannot be directly
compared because of discrepancies between the assumptions and dynamical models utilized as well as
different origins/destinations. All of the investigations besides the one from Kakoi et al. assume
that the Sun-Earth and Sun-Mars systems are coplanar and the trajectories depart from the Sun-Earth
system instead of cislunar space\cite{Topputo:2005,Haibin:2014}, resulting in a lower $\Delta v$
and TOF for those transfers by ignoring the first leg of the journey included in this investigation
and the inclination change maneuver. Kakoi's transfers target the ephemeris position of Mars
instead of terminating in a nearby periodic orbit, eliminating the need for arrival maneuvers and
manifold arcs that add $\Delta v$ and time\cite{Kakoi:2014}. Even with those simplifying
assumptions, the results from the investigations that include manifold arcs have long
times-of-flight, suggesting that the results do not contradict the findings from this
investigation.

Overall, this investigation provides transfers that have lower maneuver $\Delta v$ costs than
direct patched conic methods but with longer times-of-flight. The results are consistent with
invariant manifold arc transfers from previous literature; the maneuver costs are higher due to the
application of a more accurate dynamical model and cislunar departure orbits. While the TOF is
longer in some cases, employing transfers that directly depart the system instead of intermediate
staging in Sun-Earth halo orbits decreases those times on the order of $1$ year. Additionally, the
proposed transfer methodologies connect Earth-Moon CR3BP orbits to Sun-Mars CR3BP orbits via
invariant manifolds arcs, a likely future mission scenario that has not received a lot of attention
to date.
