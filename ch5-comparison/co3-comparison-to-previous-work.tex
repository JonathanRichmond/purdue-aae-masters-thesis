\section{Comparison to Previous Work}
As mentioned previously in Section 1.3, several authors have investigated transfers between Earth
or cislunar space and Mars. These studies us a variety of methodologies to accomplish these
transfers, leveraging direct impulsive burns, low-thrust, and/or invariant manifolds. Some of their
relevant findings are provided here for comparison to the results of this investigation.

It is important to note that each of the transfers from these investigations, whose results are
consolidated into \cref{tab:transferCosts}, uses a different methodology. Some use an optimizer,
others leverage invariant manifolds, and the number and placement of maneuvers varies from transfer
to transfer. Additionally, each one departs from and arrives at different locations, making it
difficult to directly compare the total maneuver $\Delta v$ and TOF values. Many of the
investigations also assume that the orbital planes of Earth and Mars are coplanar, neglecting the
nontrivial plane change maneuver required when considering true orbital planes. Therefore, a direct
comparison cannot be made between all of these methodologies. However, they do provide some
intuition for the costs of similar interplanetary transfers.

...

\begin{table}[ht]
    \centering
    \caption{Representative transfer costs from previous investigations.}
    \begin{tabular}{|c|c|c|c|c|}
        \hline
        \textbf{Transfer Type}                      &   \textbf{Origin} &   \textbf{Destination}    &   \boldmath$\Delta v$ \textbf{[km/s]} &   \boldmath$TOF$ \textbf{[days]}  \\  \hline
        Hohmann Transfer                            &   Earth           &   Sun-Mars $L_{1}$        &   5.64                                &   258                             \\  \hline
        Lambert Arc\cite{Eagle:2022}                &   Earth           &   Mars                    &   5.61                                &   310                             \\  \hline
        Optimized 4-Body\cite{Miele:1999}           &   LEO             &   LMO                     &   5.65                                &   258                             \\  \hline
        Lambert Arc CR3BP\cite{Conte:2017}          &   Lunar DRO       &   LMO                     &   3.29                                &   206                             \\  \hline
        CR3BP Manifolds\cite{Topputo:2005}          &   S-E Lyapunov    &   S-M Lyapunov            &   3.76                                &   999                             \\  \hline
        Pseudo-Manifolds\cite{Haibin:2014}          &   S-E Halo        &   S-M Halo                &   2.06                                &   596                             \\  \hline
        CR3BP Manifolds\cite{Kakoi:2014}            &   E-M Halo        &   Mars                    &   0.92                                &   1784                            \\  \hline
        CR3BP Manifolds\cite{Cavallari:2019}        &   Lunar DRO       &   S-M Lyapunov            &   5.40                                &   1102                            \\  \hline
        CR3BP MMAT\cite{Canales:2022}               &   S-E Halo        &   S-M Halo                &   4.53                                &   1776                            \\  \hline
    \end{tabular}
    \label{tab:transferCosts}
\end{table}

Kakoi et al. provided the inspiration for the transfer methodology between the Earth-Moon and
Sun-Earth CR3BP models\cite{Kakoi:2014,Kakoi:2015}. Their investigation focused on comparing
different transfer scenarios departing from Earth-Moon $L_{1}$ and $L_{2}$ halo orbits. While the
strategies used are slightly different, Kakoi et al. construct transfers that utilize Sun-Earth
unstable invariant manifolds and compare them to direct transfers from Earth-Moon halo orbits. As
some sample values for an $L_{2}$ halo departure, their investigation found that a transfer along a
Sun-Earth halo invariant manifold required a total maneuver cost of $1.604$ km/s and a total TOF of
$1762$ days ($4.82$ years). Note that in their investigation, a maneuver is applied along the
departure manifold arc to speed up the transfer and the final destination is Mars' location,
cutting off the long arrival time of the stable manifold arc. On the other hand, directly using the
unstable manifold from the Earth-Moon halo orbit resulted in a transfer with a $\Delta v$ of
$0.921$ km/s and a TOF of $1784$ days ($4.88$ years)\cite{Kakoi:2015}. This direct transfer again
uses a maneuver along the departure manifold arc, as well as an inclination change maneuver during
an Earth-flyby. Other direct transfers provided in the study have lower and higher times-of-flight
and maneuver costs, showing that there a wide variety of available options. This also reflects the
findings of the current investigation, where some orbits have "direct" transfers that are more
desirable than the staging orbit transfers but other orbits do not. While these values cannot be
directly compared to those from this investigation, the general conclusion of Kakoi et al. is that
the utilization of direct Earth-Moon manifolds allows for more flexible departure dates and
transfer options, while the comparison of "direct" transfers to those with Sun-Earth staging orbits
is orbit-dependent.
