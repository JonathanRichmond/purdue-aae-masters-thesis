\section{Comparison to Previous Work}
As mentioned previously in Section 1.3, several authors have investigated transfers between Earth
or cislunar space and Mars. These studies use a variety of methodologies to accomplish these
transfers, leveraging direct impulsive burns, low-thrust, and/or invariant manifolds. Some of their
relevant findings are provided here for comparison to the results of this investigation.

It is important to note that each of the transfers from these investigations, whose results are
consolidated into \cref{tab:transferCosts}, uses a different methodology. Some use an optimizer,
others leverage invariant manifolds, and the number and placement of maneuvers vary from transfer
to transfer. Additionally, each one departs from and arrives at different locations, making it
difficult to directly compare the total maneuver $\Delta v$ and TOF values. Many of the
investigations also assume that the orbital planes of Earth and Mars are coplanar, neglecting the
nontrivial plane change maneuver required when considering true orbital planes. Therefore, a direct
comparison cannot be made between all of these methodologies. However, they do provide some
intuition for the costs of similar interplanetary transfers.

\begin{table}[ht]
    \centering
    \caption{Representative transfer costs from previous and current investigations.}
    \begin{tabular}{|c|c|c|c|c|}
        \hline
        \textbf{Transfer Type}                      &   \textbf{Origin} &   \textbf{Destination}    &   \boldmath$\Delta v$ \textbf{[km/s]} &   \boldmath$TOF$ \textbf{[days]}  \\  \hline
        Hohmann Transfer                            &   Earth           &   Sun-Mars $L_{1}$        &   5.64                                &   258                             \\  \hline
        Lambert Arc\cite{Eagle:2022}                &   Earth           &   Mars                    &   5.61                                &   310                             \\  \hline
        Optimized 4-Body\cite{Miele:1999}           &   LEO             &   LMO                     &   5.65                                &   258                             \\  \hline
        Lambert Arc CR3BP\cite{Conte:2017}          &   Lunar DRO       &   LMO                     &   3.29                                &   206                             \\  \hline
        CR3BP Manifolds\cite{Topputo:2005}          &   S-E Lyapunov    &   S-M Lyapunov            &   3.76                                &   999                             \\  \hline
        Pseudo-Manifolds\cite{Haibin:2014}          &   S-E Halo        &   S-M Halo                &   2.06                                &   596                             \\  \hline
        CR3BP Manifolds\cite{Kakoi:2014}            &   E-M Halo        &   Mars                    &   0.92                                &   1784                            \\  \hline
        CR3BP Manifolds\cite{Cavallari:2019}        &   Lunar DRO       &   S-M Lyapunov            &   5.40                                &   1102                            \\  \hline
        CR3BP MMAT\cite{Canales:2022}               &   S-E Halo        &   S-M Halo                &   4.53                                &   1776                            \\  \hline
        CR3BP MMAT                                  &   E-M Halo        &   S-M Halo                &   5.10                                &   1520                            \\  \hline
    \end{tabular}
    \label{tab:transferCosts}
\end{table}

Kakoi et al. inspired the transfer methodology between the Earth-Moon and Sun-Earth CR3BP
models\cite{Kakoi:2014,Kakoi:2015}. Their investigation focused on comparing different transfer
scenarios departing from Earth-Moon $L_{1}$ and $L_{2}$ halo orbits. While the strategies used are
slightly different, Kakoi et al. construct transfers that utilize Sun-Earth unstable invariant
manifolds and compare them to direct transfers from Earth-Moon halo orbits. As some sample values
for an $L_{2}$ halo departure, their investigation found that a transfer along a Sun-Earth halo
invariant manifold required a total maneuver cost of $1.604$ km/s and a total TOF of $1762$ days
($4.82$ years). Note that in their investigation, a maneuver is applied along the departure
manifold arc to speed up the transfer and the final destination is Mars' location, cutting off the
long arrival time of the stable manifold arc. On the other hand, directly using the unstable
manifold from the Earth-Moon halo orbit resulted in a transfer with a $\Delta v$ of $0.921$ km/s
and a TOF of $1784$ days ($4.88$ years)\cite{Kakoi:2015}. This direct transfer again uses a
maneuver along the departure manifold arc, as well as an inclination change maneuver during an
Earth-flyby. Other direct transfers provided in the study have lower and higher times-of-flight and
maneuver costs, showing that there is a wide variety of available options.

The results of the current investigation are consistent with Kakoi's conclusions. Utilizing
invariant manifolds can decrease transfer maneuver $\Delta v$ values while increasing the
TOF\cite{Kakoi:2015}. This is shown here via comparison to the results of previous investigations
and direct comparison to the patched conic "Hohmann" solution. While Kakoi's transfer data cannot
be directly compared to those from this investigation, the general conclusion of Kakoi et al. is
that the utilization of direct Earth-Moon manifolds allows for more flexible departure dates and
transfer options, while the comparison of "direct" transfers to those with Sun-Earth staging orbits
is orbit-dependent\cite{Kakoi:2014,Kakoi:2015}. This investigation supports the flexibility of the
"direct" transfer methodology and concludes that the maneuver cost comparison is orbit-dependent.
However, with the MMAT strategy used to connect the planetary system manifolds, it is shown that
the "direct" options have shorter times-of-flight compared to when an intermediate staging orbit is
included.

From a direct comparison between the $\Delta v$ and TOF results of this investigation and the
previous studies summarized in \cref{tab:transferCosts}, incorporating invariant manifolds can
decrease maneuver cost on the order of $1$ km/s compared to Hohmann transfers, Lambert
arcs\cite{Eagle:2022}, and optimized direct solutions from LEO to LMO\cite{Miele:1999}. As
explained, the manifold arcs do significantly increase the transfer TOF, but this is an acceptable
trade-off for some mission scenarios and depends on the transfer design. The results are also
consistent with some previous manifold transfers between CR3BP
orbits\cite{Cavallari:2019,Canales:2022}.

At first glance, several of the entries in \cref{tab:transferCosts} that utilize manifolds appear
to have better results than those produced by this investigation. However, the transfers cannot be
directly compared because of discrepancies between the assumptions and dynamical models utilized as
well as different origins/destinations. All of these investigations besides the one from Kakoi et
al. assume that the Sun-Earth and Sun-Mars systems are coplanar and depart from the Sun-Earth
system instead of cislunar space\cite{Topputo:2005,Haibin:2014}. This results in a lower $\Delta v$
and TOF for those transfers by ignoring the first leg of the journey and the inclination change
maneuver. Kakoi's transfers target Mars' position instead of terminating in a nearby orbit,
eliminating the need for arrival maneuvers and manifold arcs that end $\Delta v$ and
time\cite{Kakoi:2014}. Even with those simplifying assumptions, the results from these
investigations that include invariant manifolds have long times-of-flight, suggesting that these
results do not contradict the findings from this investigation.

Overall, this investigation provides transfers that have lower maneuver $\Delta v$ costs than
direct patched conic methods but with longer times-of-flight. The results are consistent with
invariant manifold transfers from previous literature; the maneuver costs are higher due to the use
of a more accurate dynamical model and cislunar departure orbits. While the TOF is longer in some
cases, using transfers that directly depart the system instead of intermediate staging in Sun-Earth
halo orbits can decrease those times on the order of $1$ year. Additionally, the proposed transfer
methodologies connect Earth-Moon CR3BP orbits to Sun-Mars CR3BP orbits using invariant manifolds, a
likely future mission scenario that has not received a lot of attention to date.
