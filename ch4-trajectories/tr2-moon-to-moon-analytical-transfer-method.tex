\section{The Moon-to-Moon Analytical Transfer Method}
The MMAT method was originally created to design tours between the moons of a planet such as
Jupiter or Saturn. However, Canales also showed that it could similarly be used for interplanetary
transfers by treating the planets as "moons" of the Sun. He provides detailed derivations,
analyses, and examples of the basic MMAT strategy, as well as some extensions relevant to this
investigation\cite{Canales:2021a,Canales:2021b,Canales:2022}. More specifically, the end-to-end
cislunar-Mars transfer methodology presented here uses a distant, two-burn MMAT transfer with a
plane change. This accounts for bridging the gap between manifolds of Sun-planet CR3BP systems and
the true orbital plane inclinations of the planets.

\subsection{Methodology}
First, a departure and an arrival CR3BP arc are needed. In this investigation, the departure CR3BP
arc is either a Sun-Earth halo orbit unstable manifold trajectory or an Earth-Moon orbit unstable
manifold arc propagated with the Sun-Earth dynamics, depending on the transfer category. Either the
departure epoch or the chosen manifold arc are varied to determine the departure CR3BP arc
respectively. The arrival CR3BP arc is a Sun-Mars halo orbit stable manifold trajectory. Since an
interplanetary transfer from Earth to Mars is a outward journey, to minimize the $\Delta v$ of the
MMAT transfer, the Sun-Mars stable manifold with the smallest periapsis (relative to the Sun) is
used\cite{Canales:2021b}. (For an inward journey, say to Venus, the manifold with the largest
apoapsis is chosen.) This minimizes the Keplerian angular momentum and energy difference between
the departure and arrival CR3BP arcs.



\subsection{Example}
