\section{Invariant Manifolds}\label{sec:InvariantManifolds}
Dynamically unstable periodic orbits in the CR3BP have valuable structures called invariant
manifolds that represent a combination of the hyperbolic and oscillatory local flow near the orbit.
With applications for transfer design, trajectories along these manifold surfaces asymptotically
arrive at or depart from the periodic orbit ballistically. Perko provides the following theorem for
stable/unstable manifolds for periodic orbits\cite{Perko:1991}:
\begin{quote}
    Let $f\in C^{1}(E)$ where E is an open subset of $R^{n}$ containing a periodic orbit,
    \begin{equation}
        \Gamma:\xbar=\gamma(t),
    \end{equation}
    of $\xbardot=f(\xbar)$ of period $T$. Let $\phi_{t}$ be the flow of $\xbardot=f(\xbar)$ and
    $\gamma(t)=\phi_{t}(\xbar_{0})$. If $k$ of the characteristic exponents of $\gamma(t)$ have
    negative real part where $0\leq k\leq n-1$ and $n-k-1$ of them have positive real part then
    there is a $\delta>0$ such that the stable manifold of $\Gamma$,
    \begin{equation}
        S(\Gamma)=\{\xbar\in N_{\delta}(\Gamma)|d(\phi_{t}(\xbar),\Gamma)\rightarrow0\text{ as }t\rightarrow\infty\text{ and }\phi_{t}(\xbar)\in N_{\delta}(\Gamma)\text{ for }t\geq0\},
    \end{equation}
    is a $(k+1)$-dimensional, differentiable manifold that is positively invariant under the flow
    $\phi_{t}$ and the unstable manifold of $\Gamma$,
    \begin{equation}
        U(\Gamma)=\{\xbar\in N_{\delta}(\Gamma)|d(\phi_{t}(\xbar),\Gamma)\rightarrow0\text{ as }t\rightarrow-\infty\text{ and }\phi_{t}(\xbar)\in N_{\delta}(\Gamma)\text{ for }t\leq0\},
    \end{equation}
    is an $(n-k)$-dimensional, differentiable manifold that is negatively invariant under the flow
    $\phi_{t}$. Furthermore, the stable and unstable manifolds of $\Gamma$ intersect transversally
    in $\Gamma$.
\end{quote}
In layman's terms, if a periodic orbit is unstable (it has eigenvalues with magnitudes greater than
and lesser than $1$), then stable/unstable manifold surfaces exist that approach/leave the orbit
asymptotically.

\subsection{Approximating a Manifold Arc}
Numerical methods are essential for approximating stable and unstable manifold arcs of periodic
orbits. Since invariant manifold structures naturally depart the periodic solution asymptotically,
a deterministic maneuver is not required to transfer onto them from the orbit. However, a
consequence of the asymptotic behavior is that it is not trivial to identify the exact location
along the orbit from where a manifold trajectory departs. Consequently, a numerical process is
necessary to approximate the stable/unstable subspaces of the orbit to determine an appropriate
initial condition for an arc along the manifold surface.

Near the periodic orbit, the manifolds are tangent to their corresponding eigenspaces of the orbit.
For a selected point on the periodic orbit:
\begin{equation}
    \nubar(t_{0}+t)=\Phi(t_{0}+t,t_{0})\nubar(t_{0}),
    \label{eq:eigenvector}
\end{equation}
where $\nubar(t_{0})$ is the eigenvector at the initial condition corresponding to the stable or
unstable eigenvalue of the monodromy matrix. Once the stable/unstable eigenvector at the selected
point is obtained:
\begin{equation}
    \nubar_{n}=\frac{\nubar}{\sqrt{\nu_{x}^{2}+\nu_{y}^{2}+\nu_{z}^{2}}},
    \label{eq:normeigenvector}
\end{equation}
where $\nu_{x}$ is the $x$-component of the eigenvector and $\nubar_{n}$ is the eigenvector
normalized by the nondimensional distance from the selected point. Utilizing the normalized
stable eigenvector to approximate the initial condition for a stable manifold arc:
\begin{equation}
    \qbar^{S}=\qbar^{\Gamma}\pm d\nubar_{n}^{S},
    \label{eq:stablemanifold}
\end{equation}
where $\qbar^{\Gamma}$ is the state vector at the selected point along the periodic orbit and $d$
is a step-off distance, generally selected based on the specific CR3BP system. The scaled
eigenvector is either added to or subtracted from the periodic state as the eigenspace is
bi-directional, creating two half-manifolds for each stability. To compute an initial condition for
an unstable manifold arc, employ \cref{eq:eigenvector}-\cref{eq:stablemanifold}, replacing the
stable components for unstable ones. Stable manifold arcs are then propagated backward in time from
the initial condition (arriving at the orbit in forward time) while unstable arcs are propagated
forward in time from their initial condition. This step-off approach enables the approximation of
initial conditions for invariant manifold arcs, facilitating their propagation in the CR3BP.

As mentioned, the step-off value $d$ depends on the CR3BP system. The step-off parameter needs to
be sufficiently small to approximate the manifold structure well. However, if $d$ is too small, the
trajectory takes longer than reasonable to arrive at/depart from the orbit since the dynamics
are asymptotic\cite{Kakoi:2015}. \cref{tab:manifoldValues} provides the values of $d$ employed for
the systems in this investigation. As some examples of manifold structures,
\cref{fig:LyapunovManifold} and \cref{fig:haloManifold} illustrate the stable and unstable
manifolds for a planar Lyapunov orbit and a spatial halo orbit, respectively, in the Earth-Moon
system. Each figure also provides a single manifold arc overlayed on the full structure for
reference. A more thorough explanation of manifold theory is found in Perko\cite{Perko:1991}. These
examples illustrate how appropriate selection of the step-off value ensures accurate representation
of the manifold structures.

\begin{table}[H]
    \centering
    \caption{Manifold-related values of relevant CR3BP systems.}
    \begin{tabular}{|c|c|c|}
        \hline
        \textbf{CR3BP System}   &   \boldmath$d$ \textbf{[km]}  &   \boldmath$\Delta MI$ \textbf{Threshold} \\  \hline
        Earth-Moon              &   $25$                        &   $1\times10^{-3}$                        \\  \hline
        Sun-Earth               &   $1000$                      &   $5\times10^{-5}$                        \\  \hline
        Sun-Mars                &   $1000$                      &   $1\times10^{-5}$                        \\  \hline
    \end{tabular}
    \label{tab:manifoldValues}
\end{table}

\begin{figure}[H]
    \centering
    \includegraphics[width=0.9\textwidth]{figures/LyapunovManifold.pdf}
    \caption{Earth-Moon $L_{1}$ Lyapunov Manifolds ($C=3.05$).}
    \label{fig:LyapunovManifold}
\end{figure}

\begin{figure}[H]
    \centering
    \includegraphics[width=0.9\textwidth]{figures/HaloManifold.pdf}
    \caption{Earth-Moon $L_{2}$ Halo Manifolds ($C=3.08$).}
    \label{fig:haloManifold}
\end{figure}

\subsection{Manifold Time-of-Flight}
Since manifolds asymptotically arrive at/depart from their orbit, even with a reasonable step-off
distance, they take a long time to depart from the orbit vicinity. Under higher-fidelity dynamics,
the period of time spent near the orbit is often absorbed by the dynamics, so it is beneficial to
have a metric to determine when the manifold has departed the orbit vicinity. In this
investigation, the momentum integral, or more specifically the momentum integral difference, is
employed to indicate orbit departure:
\begin{equation}
    MI=\int_{t_{0}}^{t}(x\xdot+y\ydot+z\zdot)d\tau,
    \label{eq:momentumintegral}
\end{equation}
where all of the values are nondimensional. Starting from $MI_{0}=0$ at the step-off location, the
momentum integral is calculated for both the orbit and the manifold arc and the difference between
the two values measures the similarity of the two trajectories:
\begin{equation}
    \Delta MI=|MI-MI^{\Gamma}|.
    \label{eq:momentumdifference}
\end{equation}
Once $\Delta MI$ has reached a threshold value (determined for each system), the trajectory is
considered departed from the orbit\cite{Guzzetti:2017}. Consequently, the time it takes for the
manifold to depart is subtracted from the time-of-flight along the manifold arc starting from the
step-off. \cref{tab:manifoldValues} provides the threshold values selected for each system in this
investigation. The momentum integral difference approach ensures consistent identification of
manifold departure, enabling accurate trajectory modeling under different dynamics.
