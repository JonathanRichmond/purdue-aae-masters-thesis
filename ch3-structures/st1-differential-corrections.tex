\section{Differential Corrections}
Differential corrections are used to compute solutions in targeting problems
that satisfy provided constraints on the initial condition and the trajectory
arc. To do so, it is necessary to be able to relate downstream states to the
initial condition.

\subsection{State Transition Matrix}
The state transition matrix (STM) relates variations in an initial state,
$\qbar_{0}=\qbar(t_{0})$, to variations in a downstream state $\qbar(t)$.
Starting from a first-order Taylor series expansion about the baseline
trajectory arc, the linear variational equation is derived:
\begin{equation}
    \partial\qbardot(t)=A(t)\partial\qbar(t),
    \label{eq:linearvariational}
\end{equation}
where $A(t)$ is the Jacobian matrix for the equations of motion with respect to
the state at time $t$. A full derivation for the CR3BP $A(t)$ matrix can be
found in Zimovan, but the result is given here\cite{Zimovan:2017}:
\begin{equation}
    A(t)=\begin{bmatrix}    \frac{\partial x}{\partial x_{0}}       &   \frac{\partial x}{\partial y_{0}}       &   \frac{\partial x}{\partial z_{0}}       &   \frac{\partial x}{\partial \xdot_{0}}       &   \frac{\partial x}{\partial \ydot_{0}}       &   \frac{\partial x}{\partial \zdot_{0}}       \\
                            \frac{\partial y}{\partial x_{0}}       &   \frac{\partial y}{\partial y_{0}}       &   \frac{\partial y}{\partial z_{0}}       &   \frac{\partial y}{\partial \xdot_{0}}       &   \frac{\partial y}{\partial \ydot_{0}}       &   \frac{\partial y}{\partial \zdot_{0}}       \\
                            \frac{\partial z}{\partial x_{0}}       &   \frac{\partial z}{\partial y_{0}}       &   \frac{\partial z}{\partial z_{0}}       &   \frac{\partial z}{\partial \xdot_{0}}       &   \frac{\partial z}{\partial \ydot_{0}}       &   \frac{\partial z}{\partial \zdot_{0}}       \\
                            \frac{\partial \xdot}{\partial x_{0}}   &   \frac{\partial \xdot}{\partial y_{0}}   &   \frac{\partial \xdot}{\partial z_{0}}   &   \frac{\partial \xdot}{\partial \xdot_{0}}   &   \frac{\partial \xdot}{\partial \ydot_{0}}   &   \frac{\partial \xdot}{\partial \zdot_{0}}   \\
                            \frac{\partial \ydot}{\partial x_{0}}   &   \frac{\partial \ydot}{\partial y_{0}}   &   \frac{\partial \ydot}{\partial z_{0}}   &   \frac{\partial \ydot}{\partial \xdot_{0}}   &   \frac{\partial \ydot}{\partial \ydot_{0}}   &   \frac{\partial \ydot}{\partial \zdot_{0}}   \\
                            \frac{\partial \zdot}{\partial x_{0}}   &   \frac{\partial \zdot}{\partial y_{0}}   &   \frac{\partial \zdot}{\partial z_{0}}   &   \frac{\partial \zdot}{\partial \xdot_{0}}   &   \frac{\partial \zdot}{\partial \ydot_{0}}   &   \frac{\partial \zdot}{\partial \zdot_{0}}   \end{bmatrix}
        =\begin{bmatrix}    0                                       &   0                                       &   0                                       &   1   &   0   &   0   \\
                            0                                       &   0                                       &   0                                       &   0   &   1   &   0   \\
                            0                                       &   0                                       &   0                                       &   0   &   0   &   1   \\
                            \frac{\partial U}{\partial x\partial x} &   \frac{\partial U}{\partial x\partial y} &   \frac{\partial U}{\partial x\partial z} &   0   &   2n  &   0   \\
                            \frac{\partial U}{\partial y\partial x} &   \frac{\partial U}{\partial y\partial y} &   \frac{\partial U}{\partial y\partial z} &   -2n &   0   &   0   \\
                            \frac{\partial U}{\partial z\partial x} &   \frac{\partial U}{\partial z\partial y} &   \frac{\partial U}{\partial z\partial z} &   0   &   0   &   0   \end{bmatrix},
                            \label{eq:variationalJacobian}
\end{equation}
\begin{equation}
    \frac{\partial U}{\partial x\partial x}=1-\frac{1-\mu}{d^{3}}-\frac{\mu}{r^{3}}+\frac{3(1-\mu)(x+\mu)^{2}}{d^{5}}+\frac{3\mu(x-1+\mu)^{2}}{r^{5}},
    \label{eq:partialUpartialxx}
\end{equation}
\begin{equation}
    \frac{\partial U}{\partial x\partial y}=\frac{\partial U}{\partial y\partial x}=\frac{3(1-\mu)(x+\mu)y}{d^{5}}+\frac{3\mu(x-1+\mu)y}{r^{5}},
    \label{eq:partialUpartialxy}
\end{equation}
\begin{equation}
    \frac{\partial U}{\partial x\partial z}=\frac{\partial U}{\partial z\partial x}=\frac{3(1-\,u)(x+\mu)z}{d^{5}}+\frac{3\mu(x-1+\mu)z}{r^{5}},
    \label{eq:partialUpartialxz}
\end{equation}
\begin{equation}
    \frac{\partial U}{\partial y\partial y}=1-\frac{1-\mu}{d^{3}}-\frac{\mu}{r^{3}}+\frac{3(1-\mu)y^{2}}{d^{5}}+\frac{3\mu y^{2}}{r^{5}},
    \label{eq:partialUpartialyy}
\end{equation}
\begin{equation}
    \frac{\partial U}{\partial y\partial z}=\frac{\partial U}{\partial z\partial y}=\frac{3(1-\mu)yz}{d^{5}}+\frac{3\mu yz}{r^{5}},
    \label{eq:partialUpartialyz}
\end{equation}
\begin{equation}
    \frac{\partial U}{\partial z\partial z}=-\frac{1-\mu}{d^{3}}-\frac{\mu}{r^{3}}+\frac{3(1-\mu)z^{2}}{d^{5}}+\frac{3\mu z^{2}}{r^{5}}.
    \label{eq:partialUpartialzz}
\end{equation}

The solution to \cref{eq:linearvariational}:
\begin{equation}
    \partial\qbar(t)=\frac{\partial\qbar(t)}{\partial\qbar_{0}}\partial\qbar_{0},
    \label{eq:variationalsolution}
\end{equation}
can be rearranged to provide the STM $\Phi(t,t_{0})$:
\begin{equation}
    \Phi(t,t_{0})=\frac{\partial\qbar(t)}{\partial\qbar_{{0}}}.
    \label{eq:STM}
\end{equation}
The equation of motion for the STM can be appended to the CR3BP equations of
motion when propagating:
\begin{equation}
    \dot{\Phi}(t,t_{0})=A(t)\Phi(t,t_{0}),
    \label{eq:STMEoM}
\end{equation}
with an initial condition of $\Phi(t_{0},t_{0})=I_{6\times6}$.

\subsection{Multi-Variable Newton-Raphson Method}
Targeting problems require iterative approaches where an initial guess is
updated until it meets a set of constraints to solve a boundary value problem.
This investigation uses a multi-variable Newton-Raphson method as a
differential corrections process for single-shooting targeting problems,
applying analytical or numerical partial derivatives of constraints with
respect to the initial conditions.

If $\Xbar$ is the free variable vector and $\Fbar(\Xbar)$ is the constraint
vector, dependent on the free variables, then the goal of the targeting problem
is to find $\Xbar$ such that $\Fbar(\Xbar)=\zerobar$ (to a chosen tolerance).
Example free variable and constraint vectors will be introduced in future
sections of this chapter and document. Under the Newton-Raphson method, the
update equation is again provided by a first-order Taylor series expansion
about the initial condition $\Xbar_{0}$:
\begin{equation}
    \Fbar(\Xbar)=\Fbar(\Xbar_{0})+DF(\Xbar_{0})(\Xbar-\Xbar_{0})=\zerobar,
    \label{eq:updateequation}
\end{equation}
where $DF(\Xbar)$ is the Jacobian matrix containing the partial derivatives of
the constraint vector with respect to the free variable vector.

With this update equation, the next iteration on the initial conditions can be
computed. If the number of free variables matches the number of constraints:
\begin{equation}
    \Xbar=\Xbar_{0}-DF(\Xbar_{0})^{-1}\Fbar(\Xbar_{0}),
    \label{eq:NRsolution}
\end{equation}
and this becomes the new iteration of the initial conditions. Ideally, upon
each iteration, the norm of the constraint vector should approach closer to the
tolerance. If the number of free variables is greater than the number of
constraints, a minimum-norm solution can be used:
\begin{equation}
    \Xbar=\Xbar_{0}-DF(\Xbar_{0})^{T}(DF(\Xbar_{0})DF(\Xbar_{0})^{T})^{-1}\Fbar(\Xbar_{0}).
    \label{eq:minimumnorm}
\end{equation}
When the number of free variables is less than the number of constraints, a
least squares solution can be used, but that is not addressed in this
investigation.

\subsection{Central Difference Method}
The Newton-Raphson method uses partial derivatives to solve a targeting problem
iteratively. These partial derivatives can be provided analytically (examples
will be shown later) or numerically using an approximation method such as
central differencing. This approach is used in this investigation to check
analytical partial derivatives or when the analytical partials are overly
complicated.

The central difference method approximates the slope of the solution at
discretized points just before and after the initial condition:
\begin{equation}
    D\Fbar_{i}(\Xbar_{0})=\frac{\partial\Fbar(\Xbar_{0})}{\partial X_{i}}=\frac{\Fbar(X_{i}+\kappa)-\Fbar(X_{i}-\kappa)}{2\kappa},
    \label{eq:slope}
\end{equation}
where $X_{i}$ is one of the components of the free variable vector and $\kappa$
is a small perturbation (this investigation uses the square root of the machine
tolerance, $\sqrt{\epsilon}$). Each variable in the free variable vector is
perturbed in both directions by $\kappa$, one at a time, and the constraint
vector is evaluated at each new free variable vector and substituted into
\cref{eq:slope} above. Perturbing all of the free variables individually makes
up the numerical Jacobian matrix:
\begin{equation}
    DF(\Xbar_{0})=\begin{bmatrix}   D\Fbar_{1}(\Xbar_{0})   &   \dots   &   D\Fbar_{m}(\Xbar_{0})\end{bmatrix}.
    \label{eq:centraldifference}
\end{equation}
