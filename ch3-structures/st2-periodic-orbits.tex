\section{Periodic Orbits}
Using the multi-variable Newton-Raphson scheme described above, periodic solutions can be targeted
in a CR3BP system. In the CR3BP, these periodic solutions exist as members of families that share
similar geometric characteristics. Some of these orbit families are symmetric about a plane or axis
in the rotating frame and this information can be utilized in the targeting process. In addition,
the initial conditions for these families can be obtained through a variety of methods including
linear variational equations about the Lagrange points and bifurcations from other orbit families.

\subsection{Targeting a Planar Lyapunov Orbit}
To demonstrate the periodic orbit targeting process, the Newton-Raphson scheme will be used to
solve for a periodic orbit in the $xy$-plane of the rotating frame around the first Lagrange point.
This family of solutions is known as the $L_{1}$ Lyapunov family and they are symmetric about the
$xz$-plane. Therefore, instead of targeting the full orbit, it is only necessary to target half of
it, from one perpendicular crossing of the $xz$-plane to the next. To target one of these orbits,
starting from a desired $x_{0}$, consider the free variable vector:

\begin{equation}
    \Xbar=\begin{bmatrix}   \ydot_{0}   &   \tau    \end{bmatrix}^{T}.
    \label{eq:Lyapunovfreevar}
\end{equation}

Since the boundary value problem being solved starts from a perpendicular crossing, it is only
necessary to allow $\ydot_{0}$ to vary as the rest of the initial states will all be $0$. In
\cref{eq:Lyapunovfreevar}, $\tau$ represents the nondimensional propagation time of the initial
conditions. To target another perpendicular crossing for the endpoint of the trajectory arc, the
following constraint vector is used:

\begin{equation}
    \Fbar(\Xbar)=\begin{bmatrix}    y_{f}   &   \xdot_{f}   \end{bmatrix}^{T}=\zerobar.
    \label{eq:Lyapunovconst}
\end{equation}

The Jacobian matrix is comprised of the partial derivatives of the constaint vectore with respect
to the free variable vector, in this case, a combination of the STM and time derivatives:

\begin{equation}
    DF=\begin{bmatrix}  \frac{\partial y_{f}}{\partial\ydot_{0}}        &   \frac{\partial y_{f}}{\partial\tau}     \\
                        \frac{\partial\xdot_{f}}{\partial{\ydot_{0}}}   &   \frac{\partial\xdot_{f}}{\partial\tau}  \end{bmatrix}
      =\begin{bmatrix}  \phi_{25}                                       &   \ydot_{f}                               \\
                        \phi_{45}                                       &   \xddot_{f}                              \end{bmatrix},
\end{equation}

where $\phi$ are the corresponding elements of the STM of the propagated arc. This Jacobian matrix
can then be used with \cref{eq:NRsolution} to iteratively solve for the free variable vector
$\Xbar$ that solves the provided problem. This, combined with the given $x_{0}$ value, is the
initial state and half of the propagation time for a periodic Lyapunov orbit.
