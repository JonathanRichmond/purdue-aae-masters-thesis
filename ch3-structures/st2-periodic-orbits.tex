\section{Periodic Orbits}
With the multi-variable Newton-Raphson scheme described above, periodic solutions are targeted in a
CR3BP system. In the CR3BP, these periodic solutions exist as members of families that share
similar geometric characteristics. Some of these orbit families are symmetric about a plane or axis
in the rotating frame and this information can be utilized in the targeting process. In addition,
the initial conditions for these families can be obtained through a variety of methods including
linear variational equations about the Lagrange points and bifurcations from other orbit families.
All of the orbit families used in this investigation and shown here are selected for their relative
instability in the Earth-Moon CR3BP. For some example initial conditions and computation strategies
for orbit families, see Grebow's\cite{Grebow:2006} and Sadaka's\cite{Sadaka:2023} theses and the
NASA Jet Propulsion Laboratory CR3BP orbit database\cite{Park}.

\subsection{Lyapunov Orbits}
\phantomsection
\subsubsection{A Lyapunov Targeter}
To demonstrate the periodic orbit targeting process, a Newton-Raphson scheme is used to solve for a
periodic orbit in the $xy$-plane of the rotating frame around the first Lagrange point. This family
of solutions is known as the $L_{1}$ Lyapunov family and they are symmetric about the $xz$-plane.
Consequently, instead of targeting the full orbit, it is only necessary to target half of it, from
one perpendicular crossing of the $xz$-plane to the next. To target one of these orbits at a
specified energy level (Jacobi constant), consider the free variable vector:
\begin{equation}
    \Xbar=\begin{bmatrix}   x_{0}   &   \ydot_{0}   &   \tau    \end{bmatrix}^{T}.
    \label{eq:Lyapunovfreevar}
\end{equation}
Since the boundary value problem being solved starts from a perpendicular crossing, it is only
necessary to allow $x_{0}$ and $\ydot_{0}$ to vary as the rest of the initial states are all $0$.
In \cref{eq:Lyapunovfreevar}, $\tau$ represents the nondimensional propagation time (TOF) of the
initial conditions. To target another perpendicular crossing for the endpoint of the trajectory
arc, the following constraint vector is used:
\begin{equation}
    \Fbar(\Xbar)=\begin{bmatrix}    y_{f}   &   \xdot_{f}   &   C-C_{d} \end{bmatrix}^{T}=\zerobar,
    \label{eq:Lyapunovconst}
\end{equation}
where $C$ is the Jacobi constant of the propagated arc and $C_{d}$ is the desired Jacobi constant.
The Jacobian matrix is then comprised of partial derivatives from the STM, time derivatives, and
partial derivatives of the Jacobi constant with respect to state variables:
\begin{equation}
    DF(\Xbar)=\begin{bmatrix}   \frac{\partial y_{f}}{\partial x_{0}}                                       &   \frac{\partial y_{f}}{\partial\ydot_{0}}    &   \frac{\partial y_{f}}{\partial\tau}     \\
                                \frac{\partial\xdot_{f}}{\partial x_{0}}                                    &   \frac{\partial\xdot_{f}}{\partial\ydot_{0}} &   \frac{\partial\xdot_{f}}{\partial\tau}  \\
                                \frac{\partial(C-C_{d})}{\partial x_{0}}                                    &   \frac{\partial(C-C_{d})}{\partial\ydot_{0}} &   \frac{\partial(C-C_{d})}{\partial\tau}  \end{bmatrix}
             =\begin{bmatrix}   \phi_{21}                                                                   &   \phi_{25}                                   &   \ydot_{f}                               \\
                                \phi_{41}                                                                   &   \phi_{45}                                   &   \xddot_{f}                              \\
                                2x_{0}-\frac{2(x_{0}+\mu)(1-\mu)}{d^{3}}-\frac{2\mu(x_{0}-1+\mu)}{r^{3}}    &   -2\ydot_{0}                                 &   0                                       \end{bmatrix}.
    \label{eq:Lyapunovjacobian}
\end{equation}
This Jacobian matrix is then inserted into \cref{eq:NRsolution} to iteratively solve for the free
variable vector $\Xbar$ that solves the provided problem. This provides the initial state and half
of the propagation time for a periodic Lyapunov orbit.

\subsubsection{Lyapunov Initial Guess}
An initial guess for a Lyapunov orbit close to the Lagrange point comes from the variational
equations of motion, linearized about the equilibrium point:
\begin{equation}
    x_{0}=x_{L}+\xi,
    \label{eq:xvar}
\end{equation}
\begin{equation}
    \ydot_{0}=-\beta_{3}\xi s,
    \label{eq:ydotvar}
\end{equation}
where $\xi$ is a chosen variation from the $x$-value of the Lagrange point,
\begin{equation}
    \beta_{1}=2-\frac{\frac{\partial U}{\partial x\partial x}+\frac{\partial U}{\partial y\partial y}}{2},
    \label{eq:beta1}
\end{equation}
\vspace{1mm}
\begin{equation}
    \beta_{2}=\sqrt{-\frac{\partial U}{\partial x\partial x}\frac{\partial U}{\partial y\partial y}},
    \label{eq:beta2}
\end{equation}
\vspace{1mm}
\begin{equation}
    s=\sqrt{\beta_{1}+\sqrt{\beta_{1}^{2}+\beta_{2}^{2}}},
    \label{eq:s}
\end{equation}
\vspace{1mm}
\begin{equation}
    \beta_{3}=\frac{s^{2}+\frac{\partial U}{\partial x\partial x}}{2s}.
    \label{eq:beta3}
\end{equation}
The last part of the initial guess for the free variable vector is the half-period of the orbit
$\tau$. This is approximated by propagating the initial state guess until it reaches the $x$-axis.

\subsubsection{Converged Lyapunov Orbit}
The linear variational equations only approximate the dynamics very close to the Lagrange point.
Using $\xi=0.005$ as the initial variation in the $x$-direction from the $L_{1}$ Lagrange point in
the Earth-Moon system:
\begin{equation}
    \Xbar_{0}=\begin{bmatrix}   0.841915    &   -0.0418614  &   1.29755\end{bmatrix}^{T},
    \label{eq:Lyapunovguess}
\end{equation}
and from the guess for the initial state, $C_{d}=3.186877$. This initial free variable guess is
propagated using the CR3BP equations of motion and is represented by the dashed curve in
\cref{fig:Lyapunov}. After targeting a perpendicular crossing using the targeter described above,
the solution is propagated (for $2\tau$) to obtain the full periodic Lyapunov orbit, shown as a
closed, solid curve in \cref{fig:Lyapunov}. Note that while the energy of the converged solution
matches that of the initial guess, the $x$- and $\ydot$-values have shifted slightly.

\subsubsection{Natural Parameter Continuation}
The process described above produces a single periodic solution near the Lagrange point. To compute
more solutions (orbits) in the family, especially further away from the Lagrange point where the linear
variational equations no longer apply, converged solutions are used in a continuation scheme to
compute other family members. This investigation utilizes natural parameter continuation, where one
of the parameters of a converged solution is changed by a small amount. This new guess for an orbit
is then converged, and a new solution is obtained. This continuation process is then repeated until
the scheme reaches a natural/dynamical end or a desired orbit is reached. Natural parameters of the
orbit family include (but are not limited to) components of the initial state, the period, or the
Jacobi constant. \cref{fig:L1Lyapunov} shows a large portion of the $L_{1}$ Lyapunov family in the
Earth-Moon system, continued in Jacobi constant from the orbit in \cref{fig:Lyapunov}. Lyapunov
families also exist about $L_{2}$ and $L_{3}$ and are computed via the same process. Since $L_{2}$
Lyapunov orbits are also used in this investigation, they are shown in \cref{fig:L2Lyapunov}. This
continuation method enables the systematic exploration of orbit families, providing a comprehensive
understanding of their dynamics and characteristics within the CR3BP.

\begin{figure}[H]
    \centering
    \includegraphics[width=0.75\textwidth]{figures/Lyapunov.pdf}
    \caption{Converged periodic Lyapunov orbit in the Earth-Moon barycentric rotating frame.}
    \label{fig:Lyapunov}
\end{figure}

\subsection{Orbital Stability}
Before discussing other orbit families, it is important to introduce orbital stability as it leads
to orbit family bifurcations and another way to generate new orbit families. Orbital stability
helps describe the characteristics of the orbit and the surrounding dynamics. The stability of an
orbit also informs the best transfer design strategies to minimize the $\Delta v$ cost.
Understanding orbital stability and its consequences is crucial for identifying bifurcations,
generating new orbit families, and designing efficient transfers.

\begin{figure}[H]
    \centering
    \includegraphics[width=0.9\textwidth]{figures/L1LyapunovFamily.pdf}
    \caption{Earth-Moon $L_{1}$ Lyapunov orbit family.}
    \label{fig:L1Lyapunov}
\end{figure}

\begin{figure}[H]
    \centering
    \includegraphics[width=0.9\textwidth]{figures/L2LyapunovFamily.pdf}
    \caption{Earth-Moon $L_{2}$ Lyapunov orbit family.}
    \label{fig:L2Lyapunov}
\end{figure}

\subsubsection{The Monodromy Matrix}
The STM of one revolution of a periodic orbit in the CR3BP, $\Phi(t_{0}+\mathbb{P},t_{0})$, is
called the monodromy matrix, and discretely maps the linear growth of perturbations from the
periodic solution. Some useful properties of the monodromy matrix are that it is symplectic, it has
a determinant of $1$, and its eigenvalues occur in reciprocal pairs\cite{ZimovanSpreen:2021}. Since
the trajectory is periodic in the CR3BP, two of the eigenvalues (one pair) are always $1$, denoted
the trivial pair, and correspond to the trajectory's periodicity and membership in a family of
solutions.

The stability of the orbit is characterized by the remaining two pairs of eigenvalues. Since the
monodromy matrix is a discrete-time mapping of the flow, eigenvalues that lie within the unit
circle (magnitude less than $1$) in the complex plane are stable and those outside the unit circle
are unstable. Perturbations in the stable subspace flow back toward the orbit, while perturbations
in the unstable subspace depart the orbit. If the eigenvalues lie directly on the unit circle, then
the corresponding flow is in the center subspace and remains bounded around the orbit. When the
stability of an eigenvalue pair changes, it can indicate a change in the characteristics of orbits
within a family and sometimes leads to a bifurcation in the family, discussed later.

The overall stability of the orbit is then determined by all of the eigenvalues of its monodromy
matrix. If any of the eigenvalues are unstable (greater than $1$), then the orbit is considered
linearly unstable. Note that the existence of an unstable eigenvalue implies the existence of a
stable eigenvalue because of the reciprocal pairs. Otherwise, the orbit is considered marginally
stable and all of the eigenvalues reside on the unit circle. Throughout an orbit family, while the
stability of the members may change, the eigenvalues experience a smooth (but not necessarily
monotonic) evolution.

\subsubsection{Stability Index}
A variety of metrics exist to more succinctly portray the stability of orbits, rather than
evaluating all of the eigenvalues. One such metric is a stability index, of which there are a
variety of definitions whose usefulness vary depending on the application. In this investigation,
the focus is on just the overall stability of the orbit --- whether it is unstable or marginally
stable --- so a metric that can quickly differentiate between these behaviors suffices.
Consequently, the following definition of the stability index is employed:
\begin{equation}
    \varsigma=||\bar{\lambda}||_{\infty},
    \label{eq:stabilityindex}
\end{equation}
where $\bar{\lambda}$ is a vector of the eigenvalues of the monodromy matrix and the infinity norm
returns the magnitude of the largest (magnitude) element of the vector\cite{Power:2019}. With this
definition, $\varsigma>1$ indicates an unstable orbit and $\varsigma=1$ indicates one that is
marginally stable. Other stability index definitions are utilized by Zimovan
Spreen\cite{ZimovanSpreen:2021}.

Like the eigenvalues themselves, the evolution of the stability index over an orbit family is
smooth. \cref{fig:stability} shows the stability indices for the members of the $L_{1}$ (a) and
$L_{2}$ (b) Lyapunov families from \cref{fig:L1Lyapunov} and \cref{fig:L2Lyapunov}. A larger
stability index means that the orbit has a higher instability and perturbations will experience
faster growth. Note that all of the members of both of these families are unstable. However, there
are stability changes among individual eigenvalues within each family that can lead to
bifurcations.

\begin{figure}[H]
    \begin{subfigure}[h]{0.4\linewidth}
        \includegraphics[width=\textwidth]{figures/L1LyapunovStability.pdf}
        \caption{$L_{1}$}
    \end{subfigure}
    \hfill
    \begin{subfigure}[h]{0.4\linewidth}
        \includegraphics[width=\textwidth]{figures/L2LyapunovStability.pdf}
        \caption{$L_{2}$}
    \end{subfigure}
    \caption{Earth-Moon Lyapunov family stability index evolution.}
    \label{fig:stability}
\end{figure}

\subsubsection{Time Constant}
Another useful metric of stability is the time constant, which approximates how long it takes for a
perturbation to grow by a factor of $e$. In terms of orbit revolutions:
\begin{equation}
    \Upsilon=\frac{1}{\ln\varsigma}.
    \label{eq:timeconstant}
\end{equation}
This equation is multiplied by the period of the orbit $\mathbb{P}$ to produce the time constant in
time units. \cref{fig:timeConstant} shows the time constant evolution of both Lyapunov families in
orbit revolutions. Note that a larger time constant indicates that it takes longer for a
perturbation to grow, indicating that the orbit is less unstable, and a marginally stable orbit has
an infinite time constant. This parameter provides a valuable measure of orbital stability,
offering insights into the susceptibility of an orbit to perturbations and its long-term dynamical
behavior.

\begin{figure}[H]
    \begin{subfigure}[h]{0.4\linewidth}
        \includegraphics[width=\textwidth]{figures/L1LyapunovTimeConstant.pdf}
        \caption{$L_{1}$}
    \end{subfigure}
    \hfill
    \begin{subfigure}[h]{0.4\linewidth}
        \includegraphics[width=\textwidth]{figures/L2LyapunovTimeConstant.pdf}
        \caption{$L_{2}$}
    \end{subfigure}
    \caption{Earth-Moon Lyapunov family time constant evolution.}
    \label{fig:timeConstant}
\end{figure}

\subsubsection{Bifucations}
Within the context of the CR3BP, bifurcation theory is applied to detect changes in the orbit
stability characteristics within a family that sometimes lead to the generation of new orbit
families. These new solutions branch off from the original family and generally have different
characteristics. Zimovan Spreen provides a more thorough analysis of how bifurcation theory can be
applied to the CR3BP, so only the information most relevant to this investigation will be provided
here\cite{ZimovanSpreen:2021}.

Two main bifurcation types are relevant to the orbits used in this analysis:
\begin{itemize}
    \item \textbf{Tangent bifurcations} occur when an eigenvalue pair reaches $1$, either from the
    unit circle or the real axis. With a cyclic fold tangent bifurcation, which occurs at an
    extremum in the Jacobi constant, the stability of the eigenvalues changes, but there is no new
    family of solutions created. Pitchfork tangent bifurcations produce two new families that have
    the same stability as the original family. The last subtype, transcritical tangent
    bifurcations, produce a new family and a change in the eigenvalue stability of the original
    family.
    \item \textbf{Period-multiplying bifurcations} occur when an eigenvalue pair reaches a root of
    $1$ ($\sqrt{1}$, $\sqrt[3]{1}$, $\sqrt[m]{1}$, etc.). In general, this produces a new family
    with a period $m$ times the original, but not necessarily a change in stability. The most
    commonly seen multiple is the period-doubling bifurcation, where the pair of eigenvalues meets
    at $-1$ either from the unit circle or the real axis. This results in a change in the
    stability of the eigenvalues and a new family whose bifurcating member has double the period of
    the original bifurcating orbit.
\end{itemize}
There are other methods of detecting bifurcations beyond examining the evolution of the eigenvalues
such as Broucke stability or bifurcation diagrams that are also described by Zimovan
Spreen\cite{ZimovanSpreen:2021}.

\subsubsection{New Family Generation from Bifurcation}
To find the initial conditions of an orbit in a new bifurcated family, first the precise
bifurcating orbit (within a specified tolerance) is obtained. This is achieved through a simple
bisection algorithm. The Jacobian matrix of this bifurcating orbit has an additional nullspace
compared to the other orbits in the family since another pair of eigenvalues (besides the trivial
pair) have values of $1$. Note that for a period-multiplying bifurcation, the orbit must be
propagated for $m$ revolutions to obtain the proper Jacobian matrix. When this is the case, one of
the nullspace vectors points in the direction of continuing the old family, while the other vector
indicates a direction for the new family. This process enables the precise identification of
bifurcating orbits, providing a foundation for exploring potential new orbit families.

To identify another orbit in a new bifurcated family, a singular value decomposition (SVD) of the
monodromy matrix provides the new nullspace direction as the right singular vector of $DF$
corresponding to the new singular value of $0$. Stepping in this direction from the initial
conditions of the bifurcating orbit and correcting for a periodic solution results in a new
periodic orbit belonging to the new family. This approach is also known as pseudo-arclength
continuation but the new member is then continued using any available scheme to obtain more of the
new family.

\subsection{Halo Orbits}
\phantomsection
\subsubsection{A Halo Targeter}
Similar to Lyapunov orbits, halo orbits are symmetric about the $xz$-plane, although they are
spatial in the rotating frame and not limited to the $xy$-plane. This again allows for
targeting only half of the periodic orbit, from one perpendicular crossing to the next. Since there
is now a $z$-component to the orbits, it is helpful to introduce a new free variable and constraint
for the halo targeter:
\begin{equation}
    \Xbar=\begin{bmatrix}   x_{0}   &   z_{0}   &   \ydot_{0}   &   \tau    \end{bmatrix}^{T},
    \label{eq:halofreevar}
\end{equation}
\begin{equation}
    \Fbar(\Xbar)=\begin{bmatrix}    y_{f}   &   \xdot_{f}   &   \zdot_{f}   &   C-C_{d} \end{bmatrix}^{T}=\zerobar,
    \label{eq:haloconst}
\end{equation}
\vspace{1mm}
\begin{equation}
    DF(\Xbar)=\begin{bmatrix}   \phi_{21}                                                                   &   \phi_{23}                                               &   \phi_{25}                                   &   \ydot_{f}                               \\
                                \phi_{41}                                                                   &   \phi_{43}                                               &   \phi_{45}                                   &   \xddot_{f}                              \\
                                \phi_{61}                                                                   &   \phi_{63}                                               &   \phi_{65}                                   &   \zddot_{f}                              \\
                                2x_{0}-\frac{2(x_{0}+\mu)(1-\mu)}{d^{3}}-\frac{2\mu(x_{0}-1+\mu)}{r^{3}}    &   -\frac{2z_{0}(1-\mu)}{d^{3}}-\frac{2z_{0}\mu}{r^{3}}    &   -2\ydot_{0}                                 &   0                                       \end{bmatrix}.
    \label{eq:halojacobian}
\end{equation}
The result of this targeting provides the initial state ($y_{0}=\xdot_{0}=\zdot_{0}=0$) and half of
the propagation time for a periodic halo orbit.

\subsubsection{Converged Halo Families}
An initial guess for a halo orbit comes from one of the bifurcating orbits in a Lyapunov family. At
$C\approx3.174352$, the $L_{1}$ Lyapunov family has a tangent bifurcation where the $L_{1}$ halo
family is formed. The pseudo-arclength method for generating an initial guess is used as described
above to obtain the initial guess for the initial state, propagation time, and Jacobi constant.
Using natural parameter continuation, more of the family is produced, shown in \cref{fig:L1Halo}.
Since the Lyapunov orbit bifurcates above and below the $xy$-plane, two halves of the family are
formed. \cref{fig:L1Halo} is denoted the $L_{1}$ northern halo orbits because most of each orbit is
spent north (in the rotating frame) of the Moon. This method efficiently generates and extends the
halo families.

\begin{figure}[H]
    \centering
    \includegraphics[width=0.9\textwidth]{figures/L1HaloFamily.pdf}
    \caption{Earth-Moon $L_{1}$ northern halo orbit family.}
    \label{fig:L1Halo}
\end{figure}

\cref{fig:L2Halo} shows the $L_{2}$ northern halo family, generated in the same way as the $L_{1}$
halo orbits but from the $L_{2}$ Lyapunov family. Note that these halo families are not monotonic
in Jacobi constant. The stability indices for both families are shown in \cref{fig:haloStability}.
$L_{3}$ halo orbits also exist, but are not utilized in this investigation.

\begin{figure}[H]
    \centering
    \includegraphics[width=0.9\textwidth]{figures/L2HaloFamily.pdf}
    \caption{Earth-Moon $L_{2}$ northern halo orbit family.}
    \label{fig:L2Halo}
\end{figure}

\begin{figure}[H]
    \begin{subfigure}[h]{0.4\linewidth}
        \includegraphics[width=\textwidth]{figures/L1HaloStability.pdf}
        \caption{$L_{1}$}
    \end{subfigure}
    \hfill
    \begin{subfigure}[h]{0.4\linewidth}
        \includegraphics[width=\textwidth]{figures/L2HaloStability.pdf}
        \caption{$L_{2}$}
    \end{subfigure}
    \caption{Earth-Moon Halo family stability index evolution.}
    \label{fig:haloStability}
\end{figure}

\subsection{Butterfly Orbits}
Butterfly orbits is another name for the $P_{2}HO_{1}$ $L_{2}$ orbit family: the period-doubling
bifurcation with the smallest perilune from the $L_{2}$ halo family\cite{ZimovanSpreen:2021}.
Conveniently, the same targeter can be used for this family as with the halos above. The same
pseudo-arclength method can also be used to obtain the initial guess from the bifurcating halo,
remembering to double the period of the orbit first. Like the halo orbits, this family has a
northern and southern half of the family, depending on the halo half-family that it bifurcates
from. A portion of the southern family is shown in \cref{fig:butterfly} and the stability indices
are shown in \cref{fig:butterflyStability}.

\begin{figure}[ht]
    \centering
    \includegraphics[width=0.9\textwidth]{figures/L2ButterflyFamily.pdf}
    \caption{Earth-Moon southern butterfly family.}
    \label{fig:butterfly}
\end{figure}

\begin{figure}[ht]
    \centering
    \includegraphics[width=0.5\textwidth]{figures/L2ButterflyStability.pdf}
    \caption{Earth-Moon butterfly family stability index evolution.}
    \label{fig:butterflyStability}
\end{figure}

\subsection{Axial Orbits}
\phantomsection
\subsubsection{An Axial Targeter}
Another spatial orbit family, the axial orbits, comes from a different tangent bifurcation in the
Lyapunov families. These orbits have symmetry only about the $x$-axis; therefore, the perpendicular
crossings must lie on the $x$-axis, unlike the halo orbits:
\begin{equation}
    \Xbar=\begin{bmatrix}   x_{0}   &   \ydot_{0}   &   \zdot_{0}   &   \tau    \end{bmatrix}^{T},
    \label{eq:axialfreevar}
\end{equation}
\begin{equation}
    \Fbar(\Xbar)=\begin{bmatrix}    y_{f}   &   z_{f}   &   \xdot_{f}   &   C-C_{d} \end{bmatrix}^{T}=\zerobar,
    \label{eq:axialconst}
\end{equation}
\begin{equation}
    DF(\Xbar)=\begin{bmatrix}   \phi_{21}                                                                   &   \phi_{25}   &   \phi_{26}   &   \ydot_{f}   \\
                                \phi_{31}                                                                   &   \phi_{35}   &   \phi_{36}   &   \zdot_{f}   \\
                                \phi_{41}                                                                   &   \phi_{45}   &   \phi_{46}   &   \xddot_{f}  \\
                                2x_{0}-\frac{2(x_{0}+\mu)(1-\mu)}{d^{3}}-\frac{2\mu(x_{0}-1+\mu)}{r^{3}}    &   -2\ydot_{0} &   -2\zdot_{0} &   0           \end{bmatrix}.
    \label{eq:axialjacobian}
\end{equation}
The result of this targeting will provide the initial state ($y_{0}=z_{0}=\xdot_{0}=0$) and half of
the propagation time for a periodic axial orbit.

\subsubsection{Converged Axial Families}
From the methods used to obtain the halo families, the $L_{1}$ and $L_{2}$ axial families can also
be obtained, shown in \cref{fig:L1Axial} and \cref{fig:L2Axial} respectively. Similar to the halo
orbits, two halves to the family can be obtained by bifurcating above or below the $xy$-plane,
making these the $L_{1}$ northwest and $L_{2}$ northeast axial families. Their stability indices
follow in \cref{fig:axialStability}. Again, there is an $L_{3}$ axial family, but it is not used in
this investigation.

\begin{figure}[ht]
    \centering
    \includegraphics[width=0.9\textwidth]{figures/L1AxialFamily.pdf}
    \caption{Earth-Moon $L_{1}$ northwest axial orbit family.}
    \label{fig:L1Axial}
\end{figure}

\begin{figure}[ht]
    \centering
    \includegraphics[width=0.9\textwidth]{figures/L2AxialFamily.pdf}
    \caption{Earth-Moon $L_{2}$ northeast axial orbit family.}
    \label{fig:L2Axial}
\end{figure}

\begin{figure}[ht]
    \begin{subfigure}[h]{0.4\linewidth}
        \includegraphics[width=\textwidth]{figures/L1AxialStability.pdf}
        \caption{$L_{1}$}
    \end{subfigure}
    \hfill
    \begin{subfigure}[h]{0.4\linewidth}
        \includegraphics[width=\textwidth]{figures/L2AxialStability.pdf}
        \caption{$L_{2}$}
    \end{subfigure}
    \caption{Earth-Moon axial family stability index evolution.}
    \label{fig:axialStability}
\end{figure}

\subsection{Vertical Orbits}
\phantomsection
\subsubsection{A Vertical Targeter}
Vertical orbits benefit from double symmetry about both the $xz$- and $xy$-planes. This allows for
targeting only a quarter of the orbit, with a perpendicular crossing of the $xz$-plane on one end
of the arc and a perpendicular crossing of the $x$-axis on the other. Starting from the $xz$-plane
crossing:
\begin{equation}
    \Xbar=\begin{bmatrix}   x_{0}   &   z_{0}   &   \ydot_{0}   &   \tau    \end{bmatrix}^{T},
    \label{eq:verticalfreevar}
\end{equation}
\begin{equation}
    \Fbar(\Xbar)=\begin{bmatrix}    y_{f}   &   z_{f}   &   \xdot_{f}   &   C-C_{d} \end{bmatrix}^{T}=\zerobar,
    \label{eq:verticalconst}
\end{equation}
\begin{equation}
    DF(\Xbar)=\begin{bmatrix}   \phi_{21}                                                                   &   \phi_{23}                                               &   \phi_{25}   &   \ydot_{f}   \\
                                \phi_{31}                                                                   &   \phi_{33}                                               &   \phi_{35}   &   \zdot_{f}   \\
                                \phi_{41}                                                                   &   \phi_{43}                                               &   \phi_{45}   &   \xddot_{f}  \\
                                2x_{0}-\frac{2(x_{0}+\mu)(1-\mu)}{d^{3}}-\frac{2\mu(x_{0}-1+\mu)}{r^{3}}    &   -\frac{2z_{0}(1-\mu)}{d^{3}}-\frac{2z_{0}\mu}{r^{3}}    &   -2\ydot_{0} &   0           \end{bmatrix}.
    \label{eq:verticaljacobian}
\end{equation}
The result of this targeting will provide the initial state ($y_{0}=\xdot_{0}=\zdot_{0}=0$) at the
perpendicular crossing of the $xz$-plane (top/bottom of the orbit) and one-quarter of the
propagation time for a periodic vertical orbit.

\subsubsection{Converged Vertical Families}
The vertical orbit family bifurcates from the end of the axial family when the axial orbit
intersects itself and resembles a figure-eight. Stepping in one direction shrinks the orbits as the
family collapses down to its origin Lagrange point. The other direction expands the orbits until
they look like clam shells, demonstrated for the $L_{1}$ family in \cref{fig:L1Vertical}. Similar
behavior occurs with the $L_{2}$ vertical family in \cref{fig:L2Vertical}.
\cref{fig:verticalStability} shows the stability indices for these two families; $L_{3}$ verticals
are not used in this investigation.

\begin{figure}[ht]
    \centering
    \includegraphics[width=0.9\textwidth]{figures/L1VerticalFamily.pdf}
    \caption{Earth-Moon $L_{1}$ vertical orbit family.}
    \label{fig:L1Vertical}
\end{figure}

\begin{figure}[ht]
    \centering
    \includegraphics[width=0.9\textwidth]{figures/L2VerticalFamily.pdf}
    \caption{Earth-Moon $L_{2}$ vertical orbit family.}
    \label{fig:L2Vertical}
\end{figure}

\begin{figure}[ht]
    \begin{subfigure}[h]{0.4\linewidth}
        \includegraphics[width=\textwidth]{figures/L1VerticalStability.pdf}
        \caption{$L_{1}$}
    \end{subfigure}
    \hfill
    \begin{subfigure}[h]{0.4\linewidth}
        \includegraphics[width=\textwidth]{figures/L2VerticalStability.pdf}
        \caption{$L_{2}$}
    \end{subfigure}
    \caption{Earth-Moon vertical family stability index evolution.}
    \label{fig:verticalStability}
\end{figure}
