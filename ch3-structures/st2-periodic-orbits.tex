\section{Periodic Orbits}
Using the multi-variable Newton-Raphson scheme described above, periodic solutions can be targeted
in a CR3BP system. In the CR3BP, these periodic solutions exist as members of families that share
similar geometric characteristics. Some of these orbit families are symmetric about a plane or axis
in the rotating frame and this information can be utilized in the targeting process. In addition,
the initial conditions for these families can be obtained through a variety of methods including
linear variational equations about the Lagrange points and bifurcations from other orbit families.

\subsection{Lyapunov Orbit Family}
\subsubsection{Lyapunov Targeter}
To demonstrate the periodic orbit targeting process, the Newton-Raphson scheme will be used to
solve for a periodic orbit in the $xy$-plane of the rotating frame around the first Lagrange point.
This family of solutions is known as the $L_{1}$ Lyapunov family and they are symmetric about the
$xz$-plane. Therefore, instead of targeting the full orbit, it is only necessary to target half of
it, from one perpendicular crossing of the $xz$-plane to the next. To target one of these orbits at
a specified energy level (Jacobi constant), consider the free variable vector:
\begin{equation}
    \Xbar=\begin{bmatrix}   x_{0}   &   \ydot_{0}   &   \tau    \end{bmatrix}^{T}.
    \label{eq:Lyapunovfreevar}
\end{equation}
Since the boundary value problem being solved starts from a perpendicular crossing, it is only
necessary to allow $x_{0}$ and $\ydot_{0}$ to vary as the rest of the initial states will all be
$0$. In \cref{eq:Lyapunovfreevar}, $\tau$ represents the nondimensional propagation time (TOF) of
the initial conditions. To target another perpendicular crossing for the endpoint of the trajectory
arc, the following constraint vector is used:
\begin{equation}
    \Fbar(\Xbar)=\begin{bmatrix}    y_{f}   &   \xdot_{f}   &   C-C_{d} \end{bmatrix}^{T}=\zerobar,
    \label{eq:Lyapunovconst}
\end{equation}
where $C$ is the Jacobi constant of the propagated arc and $C_{d}$ is the desired Jacobi constant.
The Jacobian matrix is then comprised of partial derivatives from the STM, time derivatives, and
partial derivatives of the Jacobi constant with respect to state variables:
\begin{equation}
    DF(\Xbar)=\begin{bmatrix}   \frac{\partial y_{f}}{\partial x_{0}}                                       &   \frac{\partial y_{f}}{\partial\ydot_{0}}    &   \frac{\partial y_{f}}{\partial\tau}     \\
                                \frac{\partial\xdot_{f}}{\partial x_{0}}                                    &   \frac{\partial\xdot_{f}}{\partial\ydot_{0}} &   \frac{\partial\xdot_{f}}{\partial\tau}  \\
                                \frac{\partial(C-C_{d})}{\partial x_{0}}                                    &   \frac{\partial(C-C_{d})}{\partial\ydot_{0}} &   \frac{\partial(C-C_{d})}{\partial\tau}  \end{bmatrix}
             =\begin{bmatrix}   \phi_{21}                                                                   &   \phi_{25}                                   &   \ydot_{f}                               \\
                                \phi_{41}                                                                   &   \phi_{45}                                   &   \xddot_{f}                              \\
                                2x_{0}-\frac{2(x_{0}+\mu)(1-\mu)}{d^{3}}-\frac{2\mu(x_{0}-1+\mu)}{r^{3}}    &   -2\ydot_{0}                                 &   0                                       \end{bmatrix}.
\end{equation}
This Jacobian matrix can then be used with \cref{eq:NRsolution} to iteratively solve for the free
variable vector $\Xbar$ that solves the provided problem. This provides the initial state and half
of the propagation time for a periodic Lyapunov orbit.

\subsubsection{Lyapunov Initial Guess}
An initial guess for a Lyapunov orbit close to the Lagrange point can come from variational
equations of motion, linearized about the equilibrium point:
\begin{equation}
    x_{0}=x_{L}+\xi,
    \label{eq:xvar}
\end{equation}
\begin{equation}
    \ydot_{0}=-\beta_{3}\xi s,
    \label{eq:ydotvar}
\end{equation}
where $\xi$ is a chosen variation from the $x$-value of the Lagrange point,
\begin{equation}
    \beta_{1}=2-\frac{\frac{\partial U}{\partial x\partial x}+\frac{\partial U}{\partial y\partial y}}{2},
    \label{eq:beta1}
\end{equation}
\begin{equation}
    \beta_{2}=\sqrt{-\frac{\partial U}{\partial x\partial x}\frac{\partial U}{\partial y\partial y}},
    \label{eq:beta2}
\end{equation}
\begin{equation}
    s=\sqrt{\beta_{1}+\sqrt{\beta_{1}^{2}+\beta_{2}^{2}}},
    \label{eq:s}
\end{equation}
\begin{equation}
    \beta_{3}=\frac{s^{2}+\frac{\partial U}{\partial x\partial x}}{2s}.
    \label{eq:beta3}
\end{equation}

The last part of the initial guess for the free variable vector is the half-period of the orbit
$\tau$. This can be approximated by propagating the initial state guess until it reaches the
$x$-axis.



\subsection{Stability}

