\chapter{INTRODUCTION}
\section{Motivation}
Recent years have seen a rapid increase in interantional interest in missions and operations in the
lunar vicinity (cislunar space). This is evidenced by the development of infrastructure for the
National Aeronautics and Space Administration's (NASA) Gateway program, a long-term hub that will
reside in an Earth-Moon near-rectilinear halo orbit (NRHO) in the lunar vicinity\cite{Zamora:2024}.
While there are a lot of exciting opportunities for cislunar missions, periodic orbits and
dynamical structures in this region of space can also be used as stepping stones on the way to
deep-space exploration.

In its 2022 Strategic Plan, NASA outlines a goal to "Extend human presence to the Moon and on
towards Mars for sustainable long-term exploration, development, and utilization" and "Enhance
space access and services"\cite{NASA:2022s}, further supported by its "Moon to Mars
Objectives" document released that same year\cite{NASA:2022m}. These documents support the idea
that operations and infrastructure in the cislunar region can be used to support future missions to
Mars and other deep-space targets.

Unlike Earth orbits, periodic orbits and trajectories in cislunar space exist in a complex
multi-body dynamical regime that is significantly affected by gravitational forces from both the
Earth and the Moon (and in some cases the Sun). While these complexities introduce new challenges
to the trajectory design process, they also provide opportunities to leverage elements of dynamical
systems theory in ways that are not possible with standard Keplerian dynamics. These include
unstable families of periodic orbits with stable and unstable invariant manifolds that are useful
for constructing transfers between orbits. As these techniques are currently being used to design
missions in the cislunar region, similar strategies can be adapted and applied to construct
transfers between cislunar orbits and deep-space targets residing outside of the Earth region.

In the past, NASA and other space agencies have designed and executed uncrewed interplanetary
missions to Mars. However, their goal is to develop a human presence on mars in the near future.
This endeavor will require an increase in crewed and uncrewed missions to the Martian vicinity,
highlighting a need for low-cost, reliable transfer strategies. Traditionally, uncrewed
interplanetary missions depart from a low-Earth orbit (LEO) or similar Earth orbit using a direct,
high-enery transfer\cite{Drake:2009}. Not only do these solutions have a high propellant cost, but
they are also point solutions that need to be recomputed whenever one of the mission parameters
changes.

The success of the application of multi-body dynamical systems theory to cislunar transfer design
suggests that these techniques can be applied to deep-space transfers with similar performance.
Utilizing unstable periodic orbit families and invariant manifold theory allows for low-energy,
lower-cost departures from the Earth-Moon system. Once departed, patched dynamical models and
existing transfer techniques can be used to complete the construction of end-to-end transfers from
the Earth-Moon system to deep-space destinations. Unfortunately, the use of invariant manifolds
often increases trajectory time-of-flight. While this is not a desirable characteristic for manned
missions (due to radiation and other concerns), it could be an acceptable trade-off for
cargo/supply missions. Therefore, rather than focusing on optimizing the interplanetary transfer
itself, comparison of some available unstable periodic orbit families leads to insight into
time-efficient departure from the Earth-Moon system. The use of simplified dynamical models also
provides families of transfers instead of point solutions, giving flexibility to the mission design
process. 

\section{Research Objectives}
To sustain an increase in missions to Mars and other deep-space targets, a framework for lower-cost
transfers from the Earth region is necessary. One promising avenue to decrease propellant costs is
to utilize invariant manifold theory for low-energy departures from unstable, cislunar, multi-body
periodic orbits. Since the use of multi-body dynamical systems theory in cislunar mission design is
relatively new, there is a lack of knowledge about the departure characteristics for families of
unstable periodic orbits in the Earth-Moon Circular Restricted 3-Body Problem (CR3BP). Additionally,
it has been shown in previous investigations that invariant manifolds from one Sun-planet CR3BP
do not connect to manifolds in other systems in a practical amount of time\cite{Koon:2000}.
Therefore, new transfer design methodologies are needed to connect two planetary CR3BPs. Finally,
since the current standard methodologies for interplanetary transfers produce only point solutions,
it would be beneficial to have new flexible methodologies that provide families of transfers.

The goal of this investigation is to develop an end-to-end, cislunar-to-Mars transfer design
methodology that provides families of solutions in order to compare the departure characteristics
of unstable Earth-Moon CR3BP periodic orbit families. The following objectives contribute to this
goal:
\begin{enumerate}
    \item   \textbf{Establish a cislunar-to-Mars low-energy transfer design methodology that
            utilizes CR3BP invariant manifolds.} This new transfer methodology will provide a
            solution for bridging the gap between the invariant manifolds of Sun-planet CR3BP
            systems. Using CR3BP and patched models, the resulting solutions will also exist in
            families, providing flexible mission design.
    \item   \textbf{Compare the use of intermediate Sun-Earth staging orbits to direct transfers.}
            Immediately once a trajectory departs the Earth-Moon CR3BP, it will be in a region of
            space that can be modeled by the Sun-Earth CR3BP. Sun-Earth invariant manifolds may
            depart this system faster than Earth-Moon manifolds (propagated in the Sun-Earth
            model). Therefore, transfers that arrive at an intermediate Sun-Earth orbit may exhibit
            more favorable departure characteristics.
    \item   \textbf{Analyze and compare the Earth-Moon departure characteristics of various
            unstable CR3BP periodic orbit families within this framework.} The ultimate goal of
            this investigation is to compare families of unstable Earth-Moon orbits to see which
            ones provide low-energy transfers to Mars with lower times-of-flight. This will inform
            future interplanetary and deep-space mission designs and lend insight into cislunar
            departure dynamics.
\end{enumerate}
This investigation primarily investigates cislunar-Mars transfers, but the techniques and developed
methodology can be applied to many other applications involving CR3BP systems and deep-space
targets. To demonstrate the methodology, all the end-to-end cislunar-to-Mars transfers will depart
from various Earth-Moon CR3BP orbits, but arrive at the same Sun-Mars $L_{1}$ northern halo orbit.

\section{Previous Contributions}


\section{Overview of Current Work}
