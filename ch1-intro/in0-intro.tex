\chapter{INTRODUCTION}
\section{Motivation}
In recent years, there has been a rapid increase in international interest in missions and
operations in the lunar vicinity (cislunar space). This is evidenced by the development of
infrastructure for the National Aeronautics and Space Administration's (NASA) Gateway program, a
long-term hub that will reside in an Earth-Moon near-rectilinear halo orbit (NRHO) in the lunar
vicinity\cite{Zamora:2024}. While there are a lot of exciting opportunities for cislunar missions,
periodic orbits and dynamical structures in this region of space can also be used as stepping
stones on the way to deep-space exploration.

In its 2022 Strategic Plan, NASA outlines a goal to "Extend human presence to the Moon and on
towards Mars for sustainable long-term exploration, development, and utilization" and "Enhance
space access and services"\cite{NASA:2022s}, further supported by its "Moon to Mars
Objectives" document released that same year\cite{NASA:2022m}. These documents support the idea
that operations and infrastructure in the cislunar region can be used to support future missions to
Mars and other deep-space targets.

Unlike Earth orbits, periodic orbits and trajectories in cislunar space exist in a complex
multi-body dynamical regime that is significantly affected by gravitational forces from both the
Earth and the Moon (and in some cases the Sun). While these complexities introduce new challenges
to the trajectory design process, they also provide opportunities to leverage elements of dynamical
systems theory in ways that are not possible with standard Keplerian dynamics. These include
unstable families of periodic orbits with stable and unstable invariant manifolds that are useful
for constructing transfers between orbits. As these techniques are currently being used to design
missions in the cislunar region, similar strategies can be adapted and applied to construct
transfers between cislunar orbits and deep-space targets residing outside of the Earth region.

In the past, NASA and other space agencies have designed and executed uncrewed interplanetary
missions to Mars. However, their goal is to develop a human presence on Mars in the near future.
This endeavor will require an increase in crewed and uncrewed missions to the Martian vicinity,
highlighting a need for low-cost, reliable transfer strategies. Traditionally, uncrewed
interplanetary missions depart from a low Earth orbit (LEO) or similar Earth orbit using a direct,
high-energy transfer\cite{Drake:2009}. Not only do these solutions have a high propellant cost, but
they are also point solutions that need to be recomputed whenever one of the mission parameters
changes.

The success of the application of multi-body dynamical systems theory to cislunar transfer design
suggests that these techniques can be applied to deep-space transfers with similar performance.
Utilizing unstable periodic orbit families and invariant manifold theory allows for low-energy,
lower-cost departures from the Earth-Moon system. Once departed, patched dynamical models and
existing transfer techniques can be used to complete the construction of end-to-end transfers from
the Earth-Moon system to deep-space destinations. Unfortunately, the use of invariant manifolds
often increases trajectory time-of-flight. While this is not a desirable characteristic for manned
missions (due to radiation and other concerns), it could be an acceptable trade-off for
cargo/supply missions. Therefore, rather than focusing on optimizing the interplanetary transfer
itself, a comparison of some available unstable periodic orbit families leads to insight into
time-efficient departure from the Earth-Moon system. The use of simplified dynamical models also
provides families of transfers instead of point solutions, giving flexibility to the mission design
process. 

\section{Research Objectives}
To sustain an increase in missions to Mars and other deep-space targets, a framework for lower-cost
transfers from the Earth region is necessary. One promising avenue to decrease propellant costs is
to utilize invariant manifold theory for low-energy departures from unstable, cislunar, multi-body
periodic orbits. Since the use of multi-body dynamical systems theory in cislunar mission design is
relatively new, there is a lack of knowledge about the departure characteristics for families of
unstable periodic orbits in the Earth-Moon Circular Restricted 3-Body Problem (CR3BP). Additionally,
it has been shown in previous investigations that invariant manifolds from one Sun-planet CR3BP
do not connect to manifolds in other systems in a practical amount of time\cite{Koon:2000}.
Therefore, new transfer design methodologies are needed to connect two planetary CR3BPs. Finally,
since the current standard methodologies for interplanetary transfers produce only point solutions,
it would be beneficial to have new flexible methodologies that provide families of transfers.

The goal of this investigation is to develop an end-to-end, cislunar-to-Mars transfer design
methodology that provides families of solutions in order to compare the departure characteristics
of unstable Earth-Moon CR3BP periodic orbit families. The following objectives contribute to this
goal:
\begin{enumerate}
    \item   \textbf{Establish a cislunar-to-Mars low-energy transfer design methodology that
            utilizes CR3BP invariant manifolds.} This new transfer methodology will provide a
            solution for bridging the gap between the invariant manifolds of Sun-planet CR3BP
            systems. Using CR3BP and patched models, the resulting solutions will also exist in
            families, providing flexible mission design.
    \item   \textbf{Compare the use of intermediate Sun-Earth staging orbits to direct transfers.}
            Immediately once a trajectory departs the Earth-Moon CR3BP, it will be in a region of
            space that can be modeled by the Sun-Earth CR3BP. Sun-Earth invariant manifolds may
            depart this system faster than Earth-Moon manifolds (propagated in the Sun-Earth
            model). Therefore, transfers that arrive at an intermediate Sun-Earth orbit may exhibit
            more favorable departure characteristics.
    \item   \textbf{Analyze and compare the Earth-Moon departure characteristics of various
            unstable CR3BP periodic orbit families within this framework.} The ultimate goal of
            this investigation is to compare families of unstable Earth-Moon orbits to see which
            ones provide low-energy transfers to Mars with lower times-of-flight. This will inform
            future interplanetary and deep-space mission designs and lend insight into cislunar
            departure dynamics.

\end{enumerate}
This investigation primarily investigates cislunar-Mars transfers, but the techniques and developed
methodology can be applied to many other applications involving CR3BP systems and deep-space
targets. To demonstrate the methodology, all the end-to-end cislunar-to-Mars transfers will depart
from various Earth-Moon CR3BP orbits but arrive at the same Sun-Mars $L_{1}$ northern halo orbit.

\section{Previous Contributions}
As mentioned above, traditionally, missions to Mars and other deep-space targets have used direct,
impulsive transfers departing from the Earth or low Earth orbit. Porkchop plots are often used in
tandem with patched conics to quickly provide point initial guesses for various
epochs\cite{Drake:2009}. Earth flybys have also been proposed by Landau and Longuski to reduce
propellant requirements\cite{Landau:2006}, while Fritz and Turkoglu use gravity-assist maneuvers
around other bodies\cite{Fritz:2016}.

While the study of multi-body dynamics originated in the 17th century, one of the biggest
breakthroughs occurred with the development of the CR3BP by Euler in 1722\cite{BarrowGreen:1997}.
However, it took centuries before this theory was used in trajectory design. Some notable missions
that have utilized multi-body dynamics include the International Sun-Earth Explorer-3
(ISEE-3)\cite{Farquhar:1984}, Genesis\cite{Lo:2001}, ARTEMIS\cite{Woodard:2009}, and
CAPSTONE\cite{Cheetham:2021}. Currently, multi-body dynamics are also being used to design the
baseline trajectory and operations for NASA's Gateway
hub\cite{Zamora:2024,Boudad:2022,ZimovanSpreen:2022}.

Within the context of interplanetary transfers and missions to deep-space targets, although
missions designed using multi-body dynamical systems theory have yet to be flown, several authors
have developed/proposed multi-body dynamics methodologies. Miele and Wang use the dynamics of a
circular-restricted 4-body problem to numerically optimize transfers between Keplerian LEO and low
Mars orbits\cite{Miele:1999}. Going a step further, Conte starts from a lunar distant retrograde
orbit (DRO), a stable CR3BP orbit, but still uses porkchop plots to design direct transfers to low
Mars orbits\cite{Conte:2017}, while Esper and Aldrin use aero-braking to arrive into a Phobos
DRO\cite{Esper:2019}.

Interplanetary transfer design strategies utilizing multi-body invariant manifolds started to
appear around 2005 with Topputo et al., who investigated transfers between CR3BP Sun-Earth and
Sun-Mars Lyapunov orbits\cite{Topputo:2005}. Nakamiya et al. conducted a similar investigation in
2010 between Sun-Earth and Sun-Mars halo orbits in the Hills restricted 3-body problem
(HR3BP)\cite{Nakamiya:2010}. In 2014, Haibin et al. added gravity assists and pseudo-manifolds to
aid in the transfer process between Sun-Earth and Sun-Mars halo orbits in the
CR3BP\cite{Haibin:2014}, while Kakoi et al. investigated transfers from Earth-Moon CR3BP halo
orbits to Mars\cite{Kakoi:2014}. To depart from a stable CR3BP orbit, Cavallari et al. in 2019 used
Earth-Moon Lyapunov manifolds to facilitate transfers between lunar DROs and Sun-Mars Lyapunov
orbits\cite{Cavallari:2019}. Most recently in 2022, Scantamburlo et al. investigated elliptic
restricted 3-body problem (ER3BP) Lyapunov manifolds to connect Sun-Earth and
Sun-Mars\cite{Scantamburlo:2022}, as Canales et al. used a semi-analytical moon-to-moon transfer
design methodology to construct trajectories between CR3BP Sun-Earth and Sun-Mars halo
orbits\cite{Canales:2021a,Canales:2022}. Concurrently with these investigations, Lu et al. (2015)\cite{Lu:2015},
Shimane and Ho (2022)\cite{Shimane:2022}, and Singh and Negi (2024)\cite{Singh:2024} investigated
the use of low-thrust maneuvers (instead of impulsive) to connect the invariant manifolds of
Sun-planet systems\cite{Lu:2015,Shimane:2022}. Almost all of these previous studies treated the
Sun-Earth and Sun-Mars systems as coplanar, and none of them included analyses of other periodic
orbit families besides Lyapunovs and halos.

Several authors have also investigated connecting Earth-Moon to Sun-Earth orbits in their
respective CR3BP systems. Masdemont et al. go directly to CR3BP Sun-Earth libration point orbits
from Keplerian lunar orbits\cite{Masdemont:2021}. The aforementioned study by Kakoi et al. in 2014
uses a patched CR3BP model to ballistically connect Earth-Moon and Sun-Earth halo orbits in their
actual respective planes\cite{Kakoi:2014}. In 2019, Guo and Lei conducted a similar investigation
to CR3BP Sun-Earth libration point orbits\cite{Guo:2019}, whereas Pasquale et al. investigated
heliocentric escape options in 2021\cite{Pasquale:2021}. Finally, other authors have also
investigated connections in other dynamical models, such as Boudad et al. with the bi-circular
restricted 4-body problem in 2021\cite{Boudad:2021}.

\section{Overview of Current Work}
The main goal of this investigation is to compare families of unstable periodic orbits in the
Earth-Moon CR3BP to determine which ones have desirable system departure characteristics. A
transfer design methodology will be developed between the Earth-Moon and Sun-Mars CR3BP systems
that utilizes invariant manifolds to facilitate this orbit family comparison. This methodology will
consist of two transfer types: one that stages at an intermediate Sun-Earth halo orbit and one that
is direct. This new methodology will provide families of transfers instead of just point solutions,
allowing for more flexible mission designs. The resulting methodology and results can be applied to
other deep-space targets besides Mars, such as Venus or asteroids, and will provide insight into
general low-energy interplanetary transfers.

The cislunar-Mars transfer design methodology and the CR3BP Earth-Moon unstable orbit family
analysis are elaborated in the following chapters:
\begin{itemize}
	\item	\textbf{Chapter 2:} This chapter introduces the dynamical models and coordinate frames
			that are used in this investigation. The two-body problem and CR3BP are the main
			dynamical models used with a couple of patched and blended model combinations to
			describe transitions between systems. Within the CR3BP, the barycentric synodic
			rotating frame is primarily used to represent trajectories, but a Sun-centered Ecliptic
			J2000 frame best represents the interplanetary transfers in their entirety.
	\item	\textbf{Chapter 3:} Many techniques from dynamical systems theory are used in this
			investigation, and those are described here. These numerical techniques and dynamical
			structures are used in both the transfer design process and the subsequent analysis of
			the system departure characteristics. This chapter also includes examples of the
			unstable CR3BP periodic orbit families that are used throughout this investigation, as
			well as procedures for generating invariant manifolds from them.
	\item	\textbf{Chapter 4:} Two existing transfer methodologies are adapted and combined to
			form the end-to-end cislunar-Mars transfer strategy. All of the transfers use a version
			of the moon-to-moon analytical transfer (MMAT) method developed by
			Canales\cite{Canales:2021b}. If transfers use the intermediate Sun-Earth staging orbit,
			a near-ballistic transfer between the Earth-Moon orbits and the Sun-Earth halo orbit is
			constructed similarly to Kakoi's methodology\cite{Kakoi:2015}. This chapter also explains
			Lambert Arcs and how they are commonly used to design direct transfers between the
			Earth and Mars.
	\item	\textbf{Chapter 5:} With a fully developed end-to-end transfer strategy, cislunar-Mars
			trajectories can be constructed, originating at a variety of unstable Earth-Moon
			orbits. Several periodic orbit families, and orbits at different Jacobi constants
			(energy levels) within the families, will be analyzed to determine which provide
			favorable system departure characteristics compared to the others. This chapter also
			includes a comparison between the direct transfers and those that utilize intermediate
			staging orbits. It will conclude with a comparison of the transfer results to those
			from the existing literature, including Lambert arc transfers.
	\item	\textbf{Chapter 6:} A summary of the main results and conclusions of the investigation
			is presented here. The success of the research objectives is discussed, and
			recommendations for potential future work are provided.
\end{itemize}
